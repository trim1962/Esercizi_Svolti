 %\documentclass[a4paper,oneside,openany]{book}
\input{../Mod_base/base}
\input{../Mod_base/grafica}
\input{../Mod_base/matematica}
\input{../Mod_base/tabelle}
\DeclareCaptionFormat{grafico}{\textbf{Grafico \thefigure}#2#3}
\DeclareCaptionFormat{esempio}{\textbf{Esempio \thefigure}#2#3}
\newcolumntype{L}{>{$\displaystyle}l<{$}}
\newcolumntype{C}{>{$\displaystyle}c<{$}}
\newcolumntype{R}{>{$\displaystyle}r<{$}}
\newcolumntype{T}{>{\centering\arraybackslash}p{1em}} 
\newcolumntype{W}{>{\sffamily\Large $}c<{$}}
%\newcolumntype{N}[1]{>{\centering\rule[-1mm]{0pt}{4.75mm}}m{#1}}
\newcolumntype{M}[1]{>{\centering}p{#1}}
\newcommand\pilH{\rule{0pt}{2.5ex}}
\newcommand\pilD{\rule[-1ex]{0pt}{0pt}}
\newlength{\gnat}
\newlength{\gnam}
\input{../Mod_base/rifIndici}
\makeindex[options=-s ../Mod_base/oldclaudio.sti]
\input{../Mod_base/pagina}
\input{../Mod_base/date}
\input{../Mod_base/loghi}
\input{../Mod_base/unita_misura}
\input{../Mod_base/utili}
\newcommand{\function}[5]{%
  \begin{array}{@{}r<{{}}@{}c@{}c@{}l@{}}
  #1\colon & #2 & {}\to{}     & #3 \\
           & #4 & {}\mapsto{} & #5
  \end{array}}

%per le semplificazioni
\usepackage{cancel}

%Simboli logici
\usepackage{gn-logic14}
\newcommand{\tabincludegraphics}[2][]{%
  $\vcenter{\hbox{\includegraphics[#1]{#2}}}$}
\newcommand{\tabincludestandalone}[2][]{%
	$\vcenter{\hbox{\includestandalone[#1]{#2}}}$}


\usepackage{xlop}
\newcount\primeindex
\newcount\tryindex
\newif\ifprime
\newif\ifagain
\newcommand\getprime[1]{%
	\opcopy{2}{P0}%
	\opcopy{3}{P1}%
	\opcopy{5}{try}
	\primeindex=2
	\loop
	\ifnum\primeindex<#1\relax
	\testprimality
	\ifprime
	\opcopy{try}{P\the\primeindex}%
	\advance\primeindex by1
	\fi
	\opadd*{try}{2}{try}%
	\ifnum\primeindex<#1\relax
	\testprimality
	\ifprime
	\opcopy{try}{P\the\primeindex}%
	\advance\primeindex by1
	\fi
	\opadd*{try}{4}{try}%
	\fi
	\repeat
}
\newcommand\testprimality{%
	\begingroup
	\againtrue
	\global\primetrue
	\tryindex=0
	\loop
	\opidiv*{try}{P\the\tryindex}{q}{r}%
	\opcmp{r}{0}%
	\ifopeq \global\primefalse \againfalse \fi
	\opcmp{q}{P\the\tryindex}%
	\ifoplt \againfalse \fi
	\advance\tryindex by1
	\ifagain
	\repeat
	\endgroup
}
\newcommand\primedecomp[2][nil]{%
	\begingroup
	\opset{#1}%
	\opcopy{#2}{NbtoDecompose}%
	\opabs{NbtoDecompose}{NbtoDecompose}%
	\opinteger{NbtoDecompose}{NbtoDecompose}%
	\opcmp{NbtoDecompose}{0}%
	\ifopeq
	I refuse to factorize zero.
	\else
	\setbox1=\hbox{\opdisplay{operandstyle.1}%
		{NbtoDecompose}}%
		{\setbox2=\box2{}}%
		\count255=1
		\primeindex=0
		\loop
		\opcmp{NbtoDecompose}{1}\ifopneq
		\opidiv*{NbtoDecompose}{P\the\primeindex}{q}{r}%
		\opcmp{0}{r}\ifopeq
		\ifvoid2
		\setbox2=\hbox{%
			\opdisplay{intermediarystyle.\the\count255}%
			{P\the\primeindex}}%
			\else
			\setbox2=\vtop{%
				\hbox{\box2}
				\hbox{%
					\opdisplay{intermediarystyle.\the\count255}%
					{P\the\primeindex}}}
					\fi
					\opcopy{q}{NbtoDecompose}%
					\advance\count255 by1
					\setbox1=\vtop{%
						\hbox{\box1}
						\hbox{%
							\opdisplay{operandstyle.\the\count255}%
							{NbtoDecompose}}
						}%
						\else
						\advance\primeindex by1
						\fi
						\repeat
						\hbox{\box1
							\kern0.5\opcolumnwidth
							\opvline(0,0.75){\the\count255.25}
							\kern0.5\opcolumnwidth
							\box2}%
							\fi
							\endgroup
							
							
}% scomposizione in fattori primi e operazioni
\getprime{20}%
% % % % % % % % % % % % % % % % % braille

\newcommand{\mytable}[1]{%
	\enskip\begin{tabular}[t]{r|l} 
		\hline #1 \hline
	\end{tabular}\enskip}

% % % % % % % % % % % % % % % % % % % % % %

\newenvironment{truthtable}[2][3]
{\begin{tabular}{*{#1}{c}}
	\multicolumn{1}{l}{#2}\\}
	{\end{tabular}}
	\newcommand{\cport}[1]{%
		\begin{circuitikz}
			\draw (0,0) node [#1 port] {};
		\end{circuitikz}} 


%todo inizio
\usepackage[italian,colorinlistoftodos]{todonotes}
%\usepackage{todonotes}

\newcommand{\bassapriorita}[1]{\todo[color=green!40]{#1}}
\newcommand{\mediapriorita}[1]{\todo[color=blue!40]{#1}}
\newcommand{\altapriorita}[1]{\todo[color=red!40]{#1}}
%todofine




%\usepackage{imakeidx}
%\makeindex[options=-s oldclaudio.sti]
\newcommand{\prodcroce}[4]{%
	\begin{tikzpicture}[thick]
	\def\x{2.8mm}
	\def\h{2.4mm}
	\def\dist{12mm}%1cm
	\node at (0,0) {$\displaystyle \frac{#1}{#2}$};
	\node at (\dist,0) {$\displaystyle \frac{#3}{#4}$};
	\node at (2.5*\dist,0) {$#1\cdot #4=#2\cdot #3$};
	% collegamento termini
	\draw[-stealth] (\x, \h)--(\dist-\x,-\h);
	\draw[-stealth] (\x,-\h)--(\dist-\x, \h);
	\end{tikzpicture}%
}
% % % % % % % loghi
%\usepackage{guit}
%\usepackage[copyright]{ccicons}
%%\usepackage{metalogo}
% % % % % % % %
% % % % % % % % % % % % % % % % % % % %TITOLO % % % % % % % % % % % % % % % % % % % % % % 
\newcommand{\HRule}{\rule{\linewidth}{0.5mm}}
% % % % % % % % % % % % % % % % % % % % % % % % % % % % % % % % % % % % % % % % % % % % %
% % % % % % % % % % % % % % % % % % % % % % %gestione data % % % % % % % % % % % % %
%\usepackage{datetime}
%\longdate
%\settimeformat{hhmmsstime}
% % % % % % % % % % % % % % % % % % % % % % % % % % % % % % % % % % % % % %
%\usepackage{cleveref}


 \usepackage[italian,checkfiles]{minitoc}
 \usepackage{placeins} 
 %\setcounter{secnumdepth}{3}     % fino a che livello le sezioni saranno numerate
% \setcounter{tocdepth}{3}       % profondit� indice globale
% \setcounter{minitocdepth}{3}



\makeatletter
\renewcommand\frontmatter{%
	\cleardoublepage
	\@mainmatterfalse
	\pagenumbering{arabic}}
\renewcommand\mainmatter{%
	\cleardoublepage
	\@mainmattertrue}
\makeatother

\listfiles

%\usepackage{calc}




 \begin{document}
\frontmatter
\begin{titlepage}
	
	\begin{center}
		
		
		% Upper part of the page
		\input{../Mod_base/Lgrande}\\[1cm]    
		
		\textsc{\LARGE Claudio Duchi}\\[1.5cm]
		
		%\textsc{\Large Final year project}\\[0.5cm]
		
		
		% Title
		\HRule \\[0.4cm]
		{ \huge \bfseries Esercizi svolti di matematica}\\[0.4cm]
		\opt{prima}{{\bfseries PRIMO}\\[0.4cm]}
		\opt{secondo}{{\bfseries SECONDO}\\[0.4cm]}
		\opt{terzo}{{\bfseries TERZO}\\[0.4cm]}
		\opt{quarto}{{\bfseries QUARTO}\\[0.4cm]}
		\opt{extra}{{\bfseries EXTRA}\\[0.4cm]}
		\opt{grafici}{{\bfseries GRAFICI}\\[0.4cm]}
		\HRule \\[1.5cm]
		
		% Author and supervisor
		%		\begin{minipage}{0.4\textwidth}
		%		%	\begin{flushleft} \large
		%				\emph{Author:}\\
		%				John \textsc{Smith}
		%		%	\end{flushleft}
		%		\end{minipage}
		%		\begin{minipage}{0.4\textwidth}
		%			\begin{flushright} \large
		%				\emph{Supervisor:} \\
		%				Dr.~Mark \textsc{Brown}
		%			\end{flushright}
		%		\end{minipage}
		
		\vfill
		
		% Bottom of the page
		{\large \today $-$\currenttime}
		
	\end{center}
	
\end{titlepage}
\setcounter{page}{2}
\input{../Mod_base/copyright}
\dominitoc
\dominilof                  % centers title of minilof's
\dominilot
\tableofcontents 
\mtcaddchapter
\cleardoublepage
\addcontentsline{toc}{chapter}{\numberline{}\listtablename}
\listoftables
\mtcaddchapter
\adjustptc
\cleardoublepage
\addcontentsline{toc}{chapter}{\numberline{}\listfigurename}
\listoffigures
\mtcaddchapter
%\todototoc
\cleardoublepage
%\addcontentsline{toc}{chapter}{\numberline{}\listoftodos}
\todototoc
\listoftodos
\mainmatter%
\opt{prima}{% !TEX encoding = UTF-8 Unicode
% !TEX TS-program = pdflatex
% !TEX encoding = UTF-8 Unicode
% !TEX root = tabelle.primo.tex
%	\chapter{Numeri Naturali}
\label{cha:numerinaturali}
\section{$\mcd$ e $\mcm$}
\label{sec:numnatmcmmcd}
\begin{esempiot} {}{}
Trovare il $\mcd$ e il $\mcm$  di \numlist{280;350;720}
\end{esempiot}
Iniziamo a scomporre i tre numeri
\stampapuntini
\begin{center}
	\begin{forest}
		[280
		[28
		[\puntini{7}]
		[4
		[2]
		[\puntini{2}]
		]
		]
		[10
		[5]
		[2]
		]
		]
	\end{forest}
	\begin{forest}
		[350
		[10
		[5]
		[2]
		]
		[\puntini{35}
		[\puntini{7}]
		[5]
		]
		]
	\end{forest}
	\begin{forest}
		[720
		[72
		[9
		[3]
		[\puntini{3}]
		]
		[\puntini{8}
		[4
		[2]
		[2]
		]
		[2]
		]
		]
		[\puntini{10}
		[5]
		[2]
		]
		]
	\end{forest}
\end{center}
Allineo i fattori

\begin{tabular}{lcllll}
	280 & = & \puntini{$2^3$} &  & 5  & 7 \\ 
	250 & = & 2     &  &\puntini{$5^2$}  & 7 \\ 
	720 & = & $2^4$ &\puntini{$3^2$}  & 5 &  \\  
\end{tabular} 

Ottengo

\begin{tabular}{lclcl}
	$\mcd$ & = & $2\cdot 5$ &=  & 10   \\ 
	$\mcm$ & = & $\puntini{2^4}\cdot 3^2\cdot5^2\cdot \puntini{7}$ &=  & 25200  \\ 
\end{tabular} 
\begin{esempiot}{}{}
	Trovare il $\mcd$ e il $\mcm$  di \numlist{432;270;288}
\end{esempiot}
Iniziamo a scomporre i tre numeri

\begin{center}
	\begin{forest}
	[432
	[\puntini{108}
	[2]
	[54
	[6
	[3]
	[2]]
	[\puntini{9}
	[3]
	[3]
	]
	]
	]
	[4
	[2]
	[2]
	]
	]
\end{forest}
\begin{forest}
[270
[\puntini{27}
[3]
[9
[3]
[3]
]
]
[\puntini{10}
[5]
[2]
]
]	
\end{forest}
\begin{forest}
	[288
	[4
	[2]
	[2]
	]
	[\puntini{72}
	[9
	[3]
	[3]
	]
	[\puntini{8}
	[4
	[2]
	[2]
	]
	[\puntini{2}]
	]
	]
	]
\end{forest}
	\end{center}
Allineo i fattori

\begin{tabular}{lclll}
432	&=  &$\puntini{2^4}$  & $3^2$ &  \\ 
270	& = & 2 & $\puntini{3^2}$ & 5 \\ 
288	& = &$2^5$  &$3^2$  &  \\ 
\end{tabular} 

Ottengo

\begin{tabular}{lclcl}
$\mcd$	&  =& $2\cdot 3^2$ &=  & 18 \\ 
$\mcm$	& = & $2^5\cdot \puntini{3^3}\cdot5$ &=  &\puntini{1440}  \\ 
\end{tabular} 
%\nonstampapuntini
\begin{esempiot}{}{}
	Trovare il $\mcd$ e il $\mcm$  di \numlist{70;48;78}
\end{esempiot}
Iniziamo a scomporre i tre numeri
\begin{center}
	\begin{forest}
		[70
		[7]
		[10
		[5]
		[2]
		]
		]
	\end{forest}
	\begin{forest}
		[48
		[6
		[3]
		[2]
		]
		[8
		[2]
		[4
		[2]
		[2]
		]
		]
		]
	\end{forest}
	\begin{forest}
		[84
		[2]
		[42
		[7]
		[6
		[3]
		[2]
		]
		]
		]
	\end{forest}
\end{center}
Allineo i fattori

\begin{tabular}{lcllll}
70	& = & 2     &  & 5 & 7 \\ 
48	& = & $2^4$ & 3 &   &   \\ 
84	& = & $2^2$ & 3 &   &  7\\ 
\end{tabular} 

Ottengo

\begin{tabular}{lclcl}
	$\mcd$	&  =& 2 &=  & 2 \\ 
	$\mcm$	& = & $2^4\cdot \puntini{3}\cdot5\cdot 7$ &=  &\puntini{1680}  \\
\end{tabular} 
%	\input{Numeri_razinali_Ass}
%	\input{proporzioni}
%	\input{Numeri_relativi}
	%\input{potenzeproprieta}
%	\input{errorieorrori}
%	\input{geometria}
%	\input{equazioni1}
	\chapter{Polinomi}
\label{cha:polinomi}
\minitoc
\mtcskip                                % put some skip here
\minilof                                % a minilof
\mtcskip                                % put some skip here
\minilot
\section{Somme}
\label{sec:somme}
\begin{esempio}
Supponiamo di voler sommare\[ 3a+2b^2+4a-6b^2+2b\] procediamo come segue:
	 \begin{NodesList} %[margin=-3cm]
	 	\begin{align*}
	 		3a+2b^2+4a-6b^2+2b&                           \AddNode\\
	 		(3+4)a+(2-6)b^2+2b&          \AddNode\\                                       		
	 		7a-4b^2+2b&   \AddNode\\
	 		\AddNode
	 	\end{align*}
	 	\LinkNodes{individuo i simili}%
	 	%\LinkNodes{sommo i monomi simili}%
	 	\LinkNodes{$3+4$ e $2-6$}%
	  \end{NodesList}
\end{esempio}
\section{Prodotti}
\begin{esempio}
Supponiamo di avere \[3(2a-5b)-7a(2a+3b)+5(a^2+3b)\]
In questo esempio abbiamo tre moltiplicazioni di un monomio per un binomio. A destra si vedono i risultati parziali  che poi sommati, danno il risultato finale.
\begin{NodesList}
	\begin{align*}
		\overbrace{3(2a-5b)}^{1}-\overbrace{7a(2a+3b)}^{2}+\overbrace{5(a^2+3b)}^{3}&\AddNode[1]\AddNode[2]\\
		6a+&\AddNode[1]&\tag{1}\\ 
		-15b&\AddNode[2]&\\
		6a-15b-7a(2a+3b)+5(a^2+3b)&\AddNode[3]\AddNode[4]\\
		-14a^2&\AddNode[3]&\tag{2}\\    
		-21ab&\AddNode[4]&\\
		6a-15b-14a-21ab+5(a^2+3b)&\nonumber\AddNode[5]\AddNode[6]\\
		+5a^2&\AddNode[5]&\tag{3}\\
		+15b&\AddNode[6]&\\
		6a-15b-14a^2-21ab+5a^2+15b&\nonumber\AddNode[7]&\\   
		6a-9a^2-21ab&\nonumber\AddNode[7] 
	\end{align*}
	\tikzset{LabelStyle/.style = {left=0.1cm,pos=0.5,text=red,fill=white}}
	\LinkNodes[margin=2cm]{$3\cdot 2a$}%    
	\LinkNodes[margin=2cm]{$3\cdot(-5b)$}%
	\LinkNodes[margin=2cm]{$-7a\cdot(2a)$}%
	\LinkNodes[margin=2cm]{$-7a\cdot(3b)$}%
	\LinkNodes[margin=2cm]{$5\cdot(a^2)$}%
	\LinkNodes[margin=2cm]{$5\cdot(3b)$}%  
	\LinkNodes[margin=2cm]{otteniamo}% 
	\LinkNodes[margin=2cm]{sommando}% 
\end{NodesList}
\end{esempio}
\begin{esempio}
Supponiamo di avere \[2a(3a-6)-(6a^2-2b)-3a(a-2b)\]
Anche in questo esempio abbiamo tre moltiplicazioni di un monomio per un binomio. Nel secondo prodotto si nota il segno meno fuori della parentesi tonda che in pratica cambierà il segno dei termini all'interno della parentesi. A destra abbiamo  i risultati parziali delle tre moltiplicazioni.
\begin{NodesList}
	\begin{align*}
		\overbrace{2a(3a-6)}^{1}-\overbrace{(6a^2-2b)}^{2}-\overbrace{3a(a-2b)}^{3}&\AddNode[1]\AddNode[2]\\
		6a^2+&\AddNode[1]&\tag{1}\\ 
		-12a&\AddNode[2]&\\
		6a^2-12a-(6a^2-2b)-3a(a-2b)&\AddNode[3]\AddNode[4]\\
		-6a^2&\AddNode[3]&\tag{2}\\    
		+2b&\AddNode[4]&\\
		6a^2-12a-6a^2+2b-3a(a-2b)&\AddNode[5]\AddNode[6]\\
		-6a&\AddNode[5]&\tag{3}\\
		+6ab&\AddNode[6]&\\
		6a^2-12a-6a^2+2b-6a+6ab&\AddNode[7]\\   
		-18a+2b+6ab&\AddNode[7]   
	\end{align*}
	\tikzset{LabelStyle/.style ={left=0.1cm,pos=0.5,text=red,fill=white}}
	\LinkNodes[margin=2cm]{$2a\cdot 3a$}%    
	\LinkNodes[margin=2cm]{$3\cdot(-5b)$}%
	\LinkNodes[margin=2cm]{$-1\cdot(6a^2)$}%
	\LinkNodes[margin=2cm]{$-1\cdot(-2b)$}%
	\LinkNodes[margin=2cm]{$-3a\cdot(a)$}%
	\LinkNodes[margin=2cm]{$-3a\cdot(-2b)$}%  
	\LinkNodes[margin=2cm]{otteniamo}% 
	\LinkNodes[margin=2cm]{sommando}% 
\end{NodesList}
\end{esempio}
\subsection{Polinomio per polinomio}
\begin{esempio}
Supponiamo di avere \[(3a-2b)(2a-b)+(2a^2-2)(2-a)\]
 In questo esempio abbiamo due moltiplicazioni di un binomio per un binomio. A destra i passaggi parziali. Infine sommiamo  gli elementi simili e otteniamo la soluzione.
 \begin{NodesList}
 	\begin{align*}
 		\overbrace{(3a-2b)(2a-b)}^{1}+\overbrace{(2a^2-2)(2-a)}^{2}&\nonumber\AddNode[1]\AddNode[2]\AddNode[3]\AddNode[4]\\
 		6a^2&\AddNode[1]&\tag{1}\\ 
 		-3ab&\AddNode[2]&\\
 		-4ab&\AddNode[3]&\\    
 		+2b^2&\AddNode[4]&\\
 		6a^2-7ab+2b^2+(2a^2-2)(2-a)&\nonumber\AddNode[5]\AddNode[6]\AddNode[7]\AddNode[8]\\
 		4a^2&\AddNode[5]&\tag{2}\\
 		-2a^3&\AddNode[6]&\\
 		-4 &\AddNode[7]&\\   
 		2a&\AddNode[8]&\\   
 		6a^2-7ab+2b^2+4a^2-2a^3-4+2a&\nonumber\AddNode[9]\\
 		10a^2+2b^2-7ab-2a^3-4+2a&\nonumber\AddNode[9]
 	\end{align*}
 	\tikzset{LabelStyle/.style ={left=.5cm,pos=.5,text=red,fill=white}}
 	\LinkNodes[margin=2cm]{$3a\cdot 2a$}%    
 	\LinkNodes[margin=2cm]{$3a\cdot(-b)$}%
 	\LinkNodes[margin=2cm]{$-2b\cdot(2a)$}%
 	\LinkNodes[margin=2cm]{$-2b\cdot(-b)$}%
 	\LinkNodes[margin=2cm]{$2a^2\cdot(2)$}%
 	\LinkNodes[margin=2cm]{$2a^2\cdot(-a)$}%  
 	\LinkNodes[margin=2cm]{$-2\cdot 2$}% 
 	\LinkNodes[margin=2cm]{$-2\cdot -a$}% 
 	\LinkNodes[margin=2cm]{Sommando}%
 \end{NodesList}
\end{esempio}
\begin{esempio}
Supponiamo di avere \[(xy-2)[(xy-2)xy+4+2xy]-(xy-2)(x^2y^2+2xy+4)\]
In questo esempio abbiamo quattro moltiplicazioni  fra vari polinomi. A complicare le cose vi sono le regole di precedenza. A destra i vari risultati parziali. Si procede seguendo l'ordine indicato sopra l'espressione. 
	\begin{NodesList}
		\begin{align*}
			\overbrace{(xy-2)\overbrace{[\underbrace{(xy-2)xy}_{1}+4+2xy]}^{2}}^{3}-\overbrace{(xy-2)(x^2y^2+2xy+4)}^{4} &\AddNode[1]\AddNode[2]\\
			x^2y^2&\AddNode[1]\\ 
			-2xy&\AddNode[2] \\
			\overbrace{(xy-2)\overbrace{[x^2y^2-2xy+4+2xy]}^{2}}^{3}-\overbrace{(xy-2)(x^2y^2+2xy+4)}^{4} &\AddNode[3]\\
			\overbrace{(xy-2)[x^2y^2+4]}^{3}-\overbrace{(xy-2)(x^2y^2+2xy+4)}^{4} &\AddNode[3]\\
			\overbrace{(xy-2)[x^2y^2+4]}^{3}-\overbrace{(xy-2)(x^2y^2+2xy+4)}^{4} &\AddNode[4]\AddNode[5]\AddNode[6]\AddNode[7]\\
			x^3y^3&\AddNode[4]\\    
			4xy&\AddNode[5]\\
			-2x^2y^2&\AddNode[6]\\
			-8&\AddNode[7]\\
			x^3y^3+4xy-2x^2y^2-8-\overbrace{(xy-2)(x^2y^2+2xy+4)}^{4} &\AddNode[8]\AddNode[9]\AddNode[10]\AddNode[11]\AddNode[12]\AddNode[13]\\
			-x^3y^3&\AddNode[8]\\
			-2x^2y^2&\AddNode[9]\\
			-4xy&\AddNode[10]\\   
			2x^2y^2&\AddNode[11] \\ 
			4xy&\AddNode[12]\\     
			8&\AddNode[13]\\   
			x^3y^3+4xy-2x^2y^2-8-x^3y^3-2x^2y^2-4xy+2x^2y^2+4xy+8 &\AddNode[14]\\
			4xy-2x^2y^2 &\AddNode[14]
		\end{align*}
		\tikzset{LabelStyle/.style = {left=0.2cm,pos=.5,text=red,fill=white}}
		\LinkNodes[margin=0cm]{$xy\cdot xy$}%         
		\LinkNodes[margin=0cm]{$-2\cdot xy$}%
		\LinkNodes[margin=0cm]{Sommando}%
		\LinkNodes[margin=0cm]{$xy\cdot(x^2y^2)$}%
		\LinkNodes[margin=0cm]{$4\cdot xy$}%
		\LinkNodes[margin=0cm]{$-2\cdot x^2y^2$}%
		\LinkNodes[margin=0cm]{$-2\cdot +4$}%
		\LinkNodes[margin=0cm]{$(-1)\cdot xy\cdot x^2y^2$}%
		\LinkNodes[margin=0cm]{$(-1)\cdot xy\cdot 2xy$}%
		\LinkNodes[margin=0cm]{$(-1)\cdot xy\cdot 4$}%
		\LinkNodes[margin=0cm]{$(-1)\cdot (-2)\cdot x^2y^2$}%
		\LinkNodes[margin=0cm]{$(-1)\cdot (-2)\cdot 2xy$}%
		\LinkNodes[margin=0cm]{$(-1)\cdot (-2)\cdot 4$}%
		\LinkNodes[margin=0cm]{Sommando}%
	\end{NodesList}
\end{esempio}
\subsection{Quadrato del binomio}
\begin{esempio}
Supponiamo di voler calcolare il quadrato del binomio \[\left(a+2b\right)^2 \]
procediamo come segue:
%\begin{figure}
\begin{NodesList}
	\begin{align*}
		\left(a+2b\right)^2&\AddNode[1]\AddNode[2]\AddNode[3]\AddNode[4]\\
		+a^2&\AddNode[1]&\\ 
		+4b^2&\AddNode[2]&\\
		+4ab&\AddNode[3]\\
		\left(a+2b\right)^2=a^2+4b^2+4ab&\AddNode[4]
	\end{align*}
	\tikzset{LabelStyle/.style = {left=0.1cm,pos=0.5,text=red,fill=white}}
	\LinkNodes[margin=2cm]{$a\cdot a$}%    
	\LinkNodes[margin=2cm]{$2b\cdot 2b$}%
	\LinkNodes[margin=2cm]{$2\cdot a \cdot 2b$}%
	\LinkNodes[margin=2cm]{ottengo}% 
\end{NodesList}
\end{esempio}
\begin{esempio}
Supponiamo di voler calcolare il quadrato di \[ \left(2x-3y\right)^2\]
procediamo come segue:
%\begin{figure}
\begin{NodesList}
	\begin{align*}
		\left(2x-3y\right)^2&\AddNode[1]\AddNode[2]\AddNode[3]\AddNode[4]\\
		+4x^2&\AddNode[1]&\\ 
		+9y^2&\AddNode[2]&\\
		-12xy&\AddNode[3]\\
		\left(2x-3y\right)^2=4x^2+9y^2-12xy&\AddNode[4]
	\end{align*}
	\tikzset{LabelStyle/.style = {left=0.1cm,pos=0.5,text=red,fill=white}}
	\LinkNodes[margin=2cm]{$2x\cdot 2x$}%    
	\LinkNodes[margin=2cm]{$(-3y)\cdot (-3y)$}%
	\LinkNodes[margin=2cm]{$2\cdot (2x) \cdot(-3y)$}%
	\LinkNodes[margin=2cm]{ottengo}% 
\end{NodesList}
\end{esempio}
\begin{esempio}
Supponiamo di voler calcolare il quadrato di \[\left(2-z\right)^2\]
%\begin{figure}
\begin{NodesList}
	\begin{align*}
		\left(2-z\right)^2&\AddNode[1]\AddNode[2]\AddNode[3]\AddNode[4]\\
		+4&\AddNode[1]&\\ 
		+z^2&\AddNode[2]&\\
		-4z&\AddNode[3]\\
		\left(2-z\right)^2=4+z^2-4z&\AddNode[4]
	\end{align*}
	\tikzset{LabelStyle/.style = {left=0.1cm,pos=0.5,text=red,fill=white}}
	\LinkNodes[margin=2cm]{$2\cdot 2$}%    
	\LinkNodes[margin=2cm]{$(-z)\cdot (-z)$}%
	\LinkNodes[margin=2cm]{$2\cdot (2) \cdot(-z)$}%
	\LinkNodes[margin=2cm]{ottengo}% 
\end{NodesList}
\end{esempio}
\begin{esempio}
Supponiamo di voler calcolare il quadrato di \[\left(1-\dfrac{1}{2}z\right)^2\]
%\begin{figure}
\begin{NodesList}
	\begin{align*}
		\left(1-\dfrac{1}{2}z\right)^2&\AddNode[1]\AddNode[2]\AddNode[3]\AddNode[4]\\
		+1&\AddNode[1]&\\ 
		+\dfrac{1}{4}z^2&\AddNode[2]&\\[0.8cm]
		-z&\AddNode[3]\\
		\left(1-\dfrac{1}{2}z\right)^2=1+\dfrac{1}{4}z^2-z&\AddNode[4]
	\end{align*}
	\tikzset{LabelStyle/.style = {left=0.1cm,pos=0.5,text=red,fill=white}}
	\LinkNodes[margin=2cm]{$1\cdot 1$}%    
	\LinkNodes[margin=2cm]{$(-\dfrac{1}{2}z)\cdot (-\dfrac{1}{2}z)$}%
	\LinkNodes[margin=2cm]{$2\cdot (1) \cdot(-\dfrac{1}{2}z)$}%
	\LinkNodes[margin=2cm]{ottengo}% 
\end{NodesList}
\end{esempio}
\subsection{Differenza di quadrati}
\begin{esempio}
Supponiamo di voler calcolare \[(2x-3y)(2x+3y)\]
procediamo come segue
\begin{NodesList}
	\begin{align*}
		(2x-3y)(2x+3y)&\AddNode[1]\AddNode[2]\AddNode[3]\\
		+4x^2&\AddNode[1]&\\ 
		-9y^2&\AddNode[2]&\\
		%-12xy&\AddNode[3]\\
		(2x-3y)(2x+3y)=4x^2-9y^2&\AddNode[3]
	\end{align*}
	\tikzset{LabelStyle/.style = {left=0.1cm,pos=0.5,text=red,fill=white}}
	\LinkNodes[margin=2cm]{$2x\cdot 2x$}%    
	\LinkNodes[margin=2cm]{$(-)(-3y)\cdot (-3y)$}%
	%\LinkNodes[margin=2cm]{$2\cdot (2x) \cdot(-3y)$}%
	\LinkNodes[margin=2cm]{ottengo}% 
\end{NodesList}
\end{esempio}
\begin{esempio}
Supponiamo di vole calcolare \[(-4a-b)(-4a+b)\]
L'esempio non sembra una differenza di quadrati ma anche qui abbiamo un termine che mantiene il segno ed un termine che lo cambia, procediamo come segue
\begin{NodesList}
	\begin{align*}
		(-4a-b)(-4a+b)&\AddNode[1]\AddNode[2]\AddNode[3]\\
		+16a^2&\AddNode[1]&\\ 
		-b^2&\AddNode[2]&\\
		%-12xy&\AddNode[3]\\
		(-4a-b)(-4a+b)=16a^2-b^2&\AddNode[3]
	\end{align*}
	\tikzset{LabelStyle/.style = {left=0.1cm,pos=0.5,text=red,fill=white}}
	\LinkNodes[margin=2cm]{$-4x\cdot(-4x)$}%    
	\LinkNodes[margin=2cm]{$(-)(-b)\cdot (-b)$}%
	%\LinkNodes[margin=2cm]{$2\cdot (2x) \cdot(-3y)$}%
	\LinkNodes[margin=2cm]{ottengo}% 
\end{NodesList}
\end{esempio}
\begin{esempio}
Supponiamo di vole calcolare \[(a+b+c)(a+b+c)\]
L'esempio non sembra una differenza di quadrati ma anche qui abbiamo un termine che mantiene il segno ed un termine che lo cambia solo che qui non è un monomio ma un binomio, procediamo come segue:
\begin{NodesList}
	\begin{align*}
		(a+b+c)(a-b-c)&\AddNode[1]\AddNode[2]\AddNode[3]\\
		[a+(b+c)][a-(b+c)]&\AddNode[1]&\\ 
		a^2-(b+c)^2&\AddNode[2]&\\
		%-12xy&\AddNode[3]\\
		(a+b+c)(a-b-c)=a^2-b^2-c^2-2bc&\AddNode[3]
	\end{align*}
	\tikzset{LabelStyle/.style = {left=0.1cm,pos=0.5,text=red,fill=white}}
	\LinkNodes[margin=2cm]{raggruppo}%    
	\LinkNodes[margin=2cm]{applico differenza di quadrati}%
	%\LinkNodes[margin=2cm]{$2\cdot (2x) \cdot(-3y)$}%
	\LinkNodes[margin=2cm]{ottengo}% 
\end{NodesList}
\end{esempio}
\subsection{Cubo del Binomio}
\begin{esempio}
Supponiamo di vole calcolare \[(a-3b)^3\]
procediamo come segue:
\begin{NodesList}
	\begin{align*}
		(a-3b)^3&\AddNode[1]\AddNode[2]\AddNode[3]\AddNode[4]\AddNode[5]\\
		a^3&\AddNode[1]&\\ 
		-27b^3&\AddNode[2]&\\
		-9a^2b&\AddNode[3]\\
		+27ab^2&\AddNode[4]\\
		(a-3b)^2=a^3-27b^3-9a^2b+27ab^2&\AddNode[5]
	\end{align*}
	\tikzset{LabelStyle/.style = {left=0.1cm,pos=0.5,text=red,fill=white}}
	\LinkNodes[margin=2cm]{$a\cdot a\cdot a $}%    
	\LinkNodes[margin=2cm]{$(-3b)\cdot (-3b)\cdot (-3b)$}%
	\LinkNodes[margin=2cm]{$3\cdot (a)\cdot (a) \cdot(-3b)$}%
	\LinkNodes[margin=2cm]{$3\cdot (a) \cdot(-3b)\cdot(-3b)$}%
	\LinkNodes[margin=2cm]{ottengo}% 
\end{NodesList}
\end{esempio}
\subsection{Quadrato del trinomio}
\begin{esempio}
Supponiamo di vole calcolare \[(a+2b-3c)^2\]
procediamo come segue:
\begin{NodesList}
	\begin{align*}
(a+2b-3c)^2&\AddNode[1]\AddNode[2]\AddNode[3]\AddNode[4]\AddNode[5]\AddNode[6]\AddNode[7]\\
		a^2&\AddNode[1]&\\ 
		+4b^2&\AddNode[2]&\\
		+9c^2&\AddNode[3]\\
		+4ab&\AddNode[4]\\
		-6ac&\AddNode[5]\\
		-12bc&\AddNode[6]\\
		(a+2b-3c)^2=a^2+4b^2+9c^2+4ab-6ac-12bc&\AddNode[7]
	\end{align*}
	\tikzset{LabelStyle/.style = {left=0.1cm,pos=0.5,text=red,fill=white}}
	\LinkNodes[margin=2cm]{$a\cdot a $}%    
	\LinkNodes[margin=2cm]{$(2b)\cdot (2b)$}%
	\LinkNodes[margin=2cm]{$(-3c)\cdot (-3c)$}%
	\LinkNodes[margin=2cm]{$2\cdot (a)\cdot (2b)$}%
	\LinkNodes[margin=2cm]{$2\cdot (a) \cdot(-3c)$}%
	\LinkNodes[margin=2cm]{$2\cdot (2b) \cdot(-3c)$}%
	\LinkNodes[margin=2cm]{ottengo}% 
\end{NodesList}
\end{esempio}

	\chapter{Divisioni fra polinomi}
\label{cha:Divisionipolinomi}

\section{Divisioni fra monomi}
\begin{esempiot}{}{}
Le seguenti divisioni sono possibili
\begin{align*}
3x^3y^2:x^y=&3x^0y^1=3y\\
4x^5a^2b:2x^2a=&2x^3ab
\end{align*}
La seguente divisione è impossibile
\begin{align*}
x^4y^3:y^5=&x^4y^{-2}
\end{align*}
\end{esempiot}
\section{Divisione fra polinomi}
\begin{esempiot}{}{}
Supponiamo di voler fare la seguente divisione $(x^3-x^4+1):(x^2+1)$
\end{esempiot}
\begin{NodesList}
\begin{align*}
(x^3-x^4+1):(x^2+1)&\AddNode\\
&\\
(-x^4+x^3+1):(x^2+1)&\AddNode\\
&\\
\begin{minipage}[t]{0.5\textwidth}
\begin{tabular}{lllll|l}
$-x^4$& $+x^3$ & \phantom{$-x^2$} &\phantom{$-x$}  &$+1$&$ x^2+1$\\ 
\cline{6-6}&&&&&
%  \vrule height 2.5ex width 0pt $-x^4$& &$-x^2$  &  &&$-x^2+x+1$\\ 
%\cline{1-5}
%  \vrule height 2.5ex width 0pt &$+x^3$ & $+x^2$ &  &$+1$&  \\ 
% &$+x^3$& &$+x$&&  \\ 
%\cline{2-5}
%   \vrule height 2.5ex width 0pt&  &$+x^2$&$-x$&$+1$&  \\ 
%   \vrule height 2.5ex width 0pt&  &$+x^2$&&$+1$&  \\ 
%\cline{3-5}
% &  &  &$-x$&&  \\ 
\\
\end{tabular}
\end{minipage}&\AddNode\\
\begin{minipage}[t]{0.5\textwidth}
\begin{tabular}{lllll|l}
$-x^4$& $+x^3$ &\phantom{$-x^2$}  &\phantom{$-x$}  &$+1$&$ x^2+1$\\ 
\cline{6-6}&&&&&$-x^2$
%  \vrule height 2.5ex width 0pt $-x^4$& &$-x^2$  &  &&$-x^2+x+1$\\ 
%\cline{1-5}
%  \vrule height 2.5ex width 0pt &$+x^3$ & $+x^2$ &  &$+1$&  \\ 
% &$+x^3$& &$+x$&&  \\ 
%\cline{2-5}
%   \vrule height 2.5ex width 0pt&  &$+x^2$&$-x$&$+1$&  \\ 
%   \vrule height 2.5ex width 0pt&  &$+x^2$&&$+1$&  \\ 
%\cline{3-5}
% &  &  &$-x$&&  \\ 
\\
\end{tabular}
\end{minipage}&\AddNode\\
\begin{minipage}[t]{0.5\textwidth}
\begin{tabular}{lllll|l}
$-x^4$& $+x^3$ & \phantom{$-x^2$} & \phantom{$-x$} &$+1$&$ x^2+1$\\ 
\cline{6-6}
  \vrule height 2.5ex width 0pt $-x^4$& &$-x^2$  &  &&$-x^2$\\ 
\cline{1-5}
  \vrule height 2.5ex width 0pt &$+x^3$ & $+x^2$ &  &$+1$&  \\ 
% &$+x^3$& &$+x$&&  \\ 
%\cline{2-5}
%   \vrule height 2.5ex width 0pt&  &$+x^2$&$-x$&$+1$&  \\ 
%   \vrule height 2.5ex width 0pt&  &$+x^2$&&$+1$&  \\ 
%\cline{3-5}
% &  &  &$-x$&&  \\ 
\\
\end{tabular}
\end{minipage}&\AddNode\\
\begin{minipage}[t]{0.5\textwidth}
\begin{tabular}{lllll|l}
$-x^4$& $+x^3$ & \phantom{$-x^2$} & \phantom{$-x$} &$+1$&$ x^2+1$\\ 
\cline{6-6}
  \vrule height 2.5ex width 0pt $-x^4$& &$-x^2$  &  &&$-x^2+x$\\ 
\cline{1-5}
  \vrule height 2.5ex width 0pt &$+x^3$ & $+x^2$ &  &$+1$&  \\ 
% &$+x^3$& &$+x$&&  \\ 
%\cline{2-5}
%   \vrule height 2.5ex width 0pt&  &$+x^2$&$-x$&$+1$&  \\ 
%   \vrule height 2.5ex width 0pt&  &$+x^2$&&$+1$&  \\ 
%\cline{3-5}
% &  &  &$-x$&&  \\ 
\\
\end{tabular}
\end{minipage}&\AddNode\\
\begin{minipage}[t]{0.5\textwidth}
\begin{tabular}{lllll|l}
$-x^4$& $+x^3$ & \phantom{$-x^2$} & \phantom{$-x$} &$+1$&$ x^2+1$\\ 
\cline{6-6}
  \vrule height 2.5ex width 0pt $-x^4$& &$-x^2$  &  &&$-x^2+x$\\ 
\cline{1-5}
  \vrule height 2.5ex width 0pt &$+x^3$ & $+x^2$ &  &$+1$&  \\ 
 &$+x^3$& &$+x$&&  \\ 
\cline{2-5}
   \vrule height 2.5ex width 0pt&  &$+x^2$&$-x$&$+1$&  \\ 
%   \vrule height 2.5ex width 0pt&  &$+x^2$&&$+1$&  \\ 
%\cline{3-5}
% &  &  &$-x$&&  \\ 
\\
\end{tabular}
\end{minipage}&\AddNode\\
\begin{minipage}[t]{0.5\textwidth}
\begin{tabular}{lllll|l}
$-x^4$& $+x^3$ & \phantom{$-x^2$} & \phantom{$-x$} &$+1$&$ x^2+1$\\ 
\cline{6-6}
  \vrule height 2.5ex width 0pt $-x^4$& &$-x^2$  &  &&$-x^2+x+1$\\ 
\cline{1-5}
  \vrule height 2.5ex width 0pt &$+x^3$ & $+x^2$ &  &$+1$&  \\ 
 &$+x^3$& &$+x$&&  \\ 
\cline{2-5}
   \vrule height 2.5ex width 0pt&  &$+x^2$&$-x$&$+1$&  \\ 
%   \vrule height 2.5ex width 0pt&  &$+x^2$&&$+1$&  \\ 
%\cline{3-5}
% &  &  &$-x$&&  \\ 
\\
\end{tabular}
\end{minipage}&\AddNode\\
\begin{minipage}[t]{0.5\textwidth}
\begin{tabular}{lllll|l}
$-x^4$& $+x^3$ & \phantom{$-x^2$} & \phantom{$-x$} &$+1$&$ x^2+1$\\ 
\cline{6-6}
  \vrule height 2.5ex width 0pt $-x^4$& &$-x^2$  &  &&$-x^2+x+1$\\ 
\cline{1-5}
  \vrule height 2.5ex width 0pt &$+x^3$ & $+x^2$ &  &$+1$&  \\ 
 &$+x^3$& &$+x$&&  \\ 
\cline{2-5}
   \vrule height 2.5ex width 0pt&  &$+x^2$&$-x$&$+1$&  \\ 
   \vrule height 2.5ex width 0pt&  &$+x^2$&&$+1$&  \\ 
\cline{3-5}
 &  &  &$-x$&&  \\ 
\\
\end{tabular}
\end{minipage}&\AddNode\\
\end{align*}
\LinkNodes{Ordino i polinomi\\ }
\LinkNodes{\begin{minipage}{3.5cm}

Scrivo la divisione lasciando spazi vuoti dove necessario\\
\end{minipage}}%
  \LinkNodes{\begin{minipage}{3.5cm}
  
 \[\dfrac{-x^4}{x^2}=-x^2\]
  \end{minipage}
}%
 \LinkNodes{Calcolo il primo resto}%
  \LinkNodes{$\dfrac{x^3}{x^2}=x$}%
\LinkNodes{\begin{minipage}{3.5cm}
Calcolo il secondo resto
\end{minipage}}%
   \LinkNodes{$\dfrac{x^2}{x^2}=1$}%
    \LinkNodes{Calcolo l'ultimo resto}%
   \end{NodesList}
\begin{esempiot}{}{}
Supponiamo di voler dividere \[(x^4+2x+1):(x^2+1)\]
\end{esempiot}
\begin{figure}
\rotatebox{90}{
\begin{minipage}[b]{.35\linewidth}
\centering\includestandalone[width=5.5cm]{primo/polinomi/divpolinomi7}
\subcaption{Sette}\label{fig:divpol2g}
\end{minipage}%
}
\rotatebox{90}{
\begin{minipage}[b]{.35\linewidth}
\centering\includestandalone[width=5.5cm]{primo/polinomi/divpolinomi6}
\subcaption{Sei}\label{fig:divpol2f}
\end{minipage}%
}
\rotatebox{90}{
\begin{minipage}[b]{.35\linewidth}
\centering\includestandalone[width=5.5cm]{primo/polinomi/divpolinomi5}
\subcaption{Cinque}\label{fig:divpol2e}
\end{minipage}%
}
\rotatebox{90}{
\begin{minipage}[b]{.35\linewidth}
\centering\includestandalone[width=5.5cm]{primo/polinomi/divpolinomi4}
\subcaption{Quattro}\label{fig:divpol2d}
\end{minipage}%
}
\rotatebox{90}{
\begin{minipage}[b]{.35\linewidth}
\centering\includestandalone[width=5.5cm]{primo/polinomi/divpolinomi3}
\subcaption{Tre}\label{fig:divpol2c}
\end{minipage}
}
\rotatebox{90}{
\begin{minipage}[b]{.35\linewidth}
\centering\includestandalone[width=5.5cm]{primo/polinomi/divpolinomi2}
\subcaption{Due}\label{fig:divpol2b}
\end{minipage}
}
\rotatebox{90}{
\begin{minipage}[b]{.35\linewidth}
\centering\includestandalone[width=5.5cm]{primo/polinomi/divpolinomi1}
\subcaption{Uno}\label{fig:divpol2a}
\end{minipage}%
}
\caption{Divisione fra polinomi}\label{fig:divpolinomi2}
\end{figure}
\section{Metodo di Ruffini}
\begin{esempiot}{}{}
Supponiamo di voler dividere
\[(x^2+2x+1):(x+1)\]
la risposta è \[x+1\] con resto zero.
\end{esempiot}
\begin{figure}
	\begin{subfigure}[b]{0.55\linewidth}
		\centering\includestandalone[width=0.6\textwidth]{primo/polinomi/ruffini1}
		\caption{Imposto il castello}\label{fig:Ruffiniesempio1a}
	\end{subfigure}%
	\captionsetup{skip=0pt}
	\begin{subfigure}[b]{0.55\linewidth}
		\centering\centering\includestandalone[width=0.6\textwidth]{primo/polinomi/ruffini2}
		\caption{Sposto il coefficiente sotto la riga}\label{fig:Ruffiniesempio1b}
	\end{subfigure}
	\begin{subfigure}[b]{.55\linewidth}
		\centering\centering\includestandalone[width=0.6\linewidth]{primo/polinomi/ruffini3}
		\caption{moltiplico e sposto}\label{fig:Ruffiniesempio1c}
	\end{subfigure}%
		\captionsetup{skip=0pt}
	\begin{subfigure}[b]{.55\linewidth}
		\centering\centering\includestandalone[width=0.6\textwidth]{primo/polinomi/ruffini4}
		\caption{Sommo sulla colonna}\label{fig:Ruffiniesempio1d}
	\end{subfigure}
	\begin{subfigure}[b]{.55\linewidth}
			\centering\centering\includestandalone[width=0.6\textwidth]{primo/polinomi/ruffini5}
			\caption{Moltiplico e sposto la risposta}\label{fig:Ruffiniesempio1e}
		\end{subfigure}%
		\captionsetup{skip=0pt}
		\begin{subfigure}[b]{.55\linewidth}
			\centering\centering\includestandalone[width=0.6\textwidth]{primo/polinomi/ruffini6}
			\caption{Sommo sulla colonna fine}\label{fig:Ruffiniesempio1f}
		\end{subfigure}
	\captionof{figure}{Metodo di Ruffini}
\label{fig:Ruffiniesempio1}
\end{figure}


	\backmatter
	\cleardoublepage
	\appendix
	}
\opt{secondo}{    \input{scomposizionipoli}
	\input{semplificazioni}
	\chapter{Radicali}
\section{Prodotti}
Calcola il prodotto fra questi radicali:
\tcbstartrecording
\begin{exercise}
$\sqrt[3]{a^4b}\cdot\sqrt[4]{ab^2}$
	\tcblower
	Il prodotto è:
\begin{NodesList}
	\begin{align*}
		&\csqrt{3}{a^4b}\cdot\csqrt{4}{ab^2}\AddNode\\ &\\
		&\sqrt[12]{a^{16}b^4}\cdot\sqrt[12]{a^3b^6}\AddNode\\ &\\
		&\sqrt[12]{a^{19}b^{10}}\AddNode\\ &\\
		&\sqrt[12]{a^{19}b^{10}}\AddNode\\ &\\
	\end{align*}
	\LinkNodes[margin=4 cm]{\begin{minipage}{2cm}{Riduco allo stesso indice}
		\end{minipage}}
		\LinkNodes[margin=4 cm]{\begin{minipage}{2cm}
				Moltiplico
			\end{minipage}}
			\LinkNodes[margin=4 cm]{\begin{minipage}{3cm}
					$mcd(12,19,10)=1$
				\end{minipage}}
			
						\end{NodesList}
\end{exercise}
\begin{exercise}
	$\sqrt[8]{\dfrac{2a^3(a+b)}{3}}\cdot\sqrt[4]{\dfrac{3}{4}\dfrac{a}{b(a+b)}}$
	\tcblower
	Il prodotto è:
	\begin{NodesList}
		\begin{align*}
			&\csqrt{8}{\dfrac{2a^3(a+b)}{3}}\cdot\csqrt{4}{\dfrac{3}{4}\dfrac{a}{b(a+b)}}\AddNode\\
			&\\
			&\sqrt[8]{\dfrac{2a^3(a+b)}{3}}\cdot\sqrt[8]{\dfrac{9}{16}\dfrac{a^2}{b^2(a+b)^2}}\AddNode\\
			&\\
			&\sqrt[8]{\dfrac{18}{16}\dfrac{a^5(a+b)}{b^2(a+b)^2}}\AddNode\\
			&\\
			&\sqrt[8]{\dfrac{9}{8}\dfrac{a^5}{b^2(a+b)}}\AddNode\\
		\end{align*}
		\LinkNodes[margin=4 cm]{\begin{minipage}{2cm}{Riduco allo stesso indice}
			\end{minipage}}
			\LinkNodes[margin=4 cm]{\begin{minipage}{2cm}
					Moltiplico
				\end{minipage}}
				\LinkNodes[margin=4 cm]{\begin{minipage}{3cm}
						Semplifico
					\end{minipage}}
					
				\end{NodesList}
			\end{exercise}
			
\begin{exercise}
	$\sqrt[12]{ab^3}\cdot\sqrt[15]{\dfrac{1}{ab(a+b)}}$
	\tcblower
	Il prodotto è:
	\begin{NodesList}
		\begin{align*}
		&\csqrt{12}{ab^3}\cdot\csqrt{15}{\dfrac{1}{ab(a+b)}}\AddNode\\
		&\\
		&\sqrt[60]{a^5b^{15}}\cdot\sqrt[60]{\dfrac{1}{a^4b^4(a+b)^4}}\AddNode\\
		&\\
		&\sqrt[60]{\dfrac{a^5b^{15}}{a^4b^4(a+b)^4}}\AddNode\\
		&\\
		&\sqrt[60]{\dfrac{ab^{9}}{(a+b)^4}}\AddNode\\
		\end{align*}
		\LinkNodes[margin=4 cm]{\begin{minipage}{2cm}{Riduco allo stesso indice}
			\end{minipage}}
			\LinkNodes[margin=4 cm]{\begin{minipage}{2cm}
					Moltiplico
				\end{minipage}}
				\LinkNodes[margin=4 cm]{\begin{minipage}{3cm}
						Semplifico
					\end{minipage}}
					
				\end{NodesList}
			\end{exercise}
\begin{exercise}
	$\sqrt{c}\cdot\sqrt[3]{a+b}$
	\tcblower
	Il prodotto è:
	\begin{NodesList}
		\begin{align*}
		&\sqrt{c}\cdot\csqrt{3}{a+b}\AddNode\\
		&\\
		&\sqrt[6]{c^3}\cdot\sqrt[6]{(a+b)^2}\AddNode\\
		&\\
		&\sqrt[6]{c^3(a+b)^2}\AddNode\\
		&\\
%		&\sqrt[60]{\dfrac{ab^{9}}{(a+b)^4}}\AddNode\\
		\end{align*}
		\LinkNodes[margin=4 cm]{\begin{minipage}{2cm}{Riduco allo stesso indice}
			\end{minipage}}
			\LinkNodes[margin=4 cm]{\begin{minipage}{2cm}
					Moltiplico
				\end{minipage}}
%				\LinkNodes[margin=4 cm]{\begin{minipage}{3cm}
%						Semplifico
%					\end{minipage}}
					
				\end{NodesList}
			\end{exercise}		
\begin{exercise}
	$\sqrt{c}\cdot\sqrt[2]{a+b}$
	\tcblower
	Il prodotto è:
	\begin{NodesList}
		\begin{align*}
		&\sqrt{c}\cdot\sqrt[2]{a+b}\AddNode\\
		&\\
		&\sqrt{c(a+b)}\AddNode\\
		&\\
		&\sqrt[2]{ac+bc}\AddNode\\
		%		&\sqrt[60]{\dfrac{ab^{9}}{(a+b)^4}}\AddNode\\
		\end{align*}
		\LinkNodes[margin=4 cm]{\begin{minipage}{2cm}{Moltiplico}
			\end{minipage}}
			\LinkNodes[margin=4 cm]{\begin{minipage}{2cm}
					Semplifico a piacere
				\end{minipage}}
				%				\LinkNodes[margin=4 cm]{\begin{minipage}{3cm}
				%						Semplifico
				%					\end{minipage}}
				
			\end{NodesList}
		\end{exercise}					
%\begin{exercise}[no solution]
%	It holds:
%	\begin{equation*}
%	\frac{d}{dx}\left(\ln|x|\right) = \frac{1}{x}.
%	\end{equation*}
%\end{exercise}

\tcbstoprecording
\newpage
\section{Soluzioni Prodotti}
\tcbinputrecords
\newpage
\section{Divisioni}
Calcola il quoziente fra questi radicali:
\tcbstartrecording
\begin{exercise}
	$\sqrt[3]{a}\div\sqrt[3]{b}$
	\tcblower
	La divisione è
	\begin{NodesList}
		\begin{align*}
		&\sqrt{3}{a}\div\sqrt{3}{b}\AddNode\\
		&\\
		&\sqrt[3]{a}\cdot\sqrt[3]{\dfrac{1}{b}}\AddNode\\
		&\\
		&\sqrt[3]{\dfrac{a}{b}}\AddNode
		\end{align*}
		\LinkNodes[margin=4 cm]{\begin{minipage}{2cm}{Trasformo in moltiplicazione}
			\end{minipage}}
			\LinkNodes[margin=4 cm]{\begin{minipage}{2cm}
					Stesso indice moltiplico
				\end{minipage}}
%				\LinkNodes[margin=4 cm]{\begin{minipage}{3cm}
%						$mcd(12,19,10)=1$
%					\end{minipage}}
					
				\end{NodesList}
			\end{exercise}
			
\begin{exercise}
	$\sqrt[3]{\dfrac{a}{b}}\div\sqrt[3]{\dfrac{c}{d}}$
	\tcblower
	La divisione è
	\begin{NodesList}
		\begin{align*}
		&\sqrt[3]{\dfrac{a}{b}}\div\sqrt[3]{\dfrac{c}{d}}\AddNode\\
		&\\
		&\sqrt[3]{\dfrac{a}{b}}\cdot\sqrt[3]{\dfrac{d}{c}}\AddNode\\
		&\\
		&\sqrt[3]{\dfrac{a}{b}\cdot\dfrac{d}{c}}\AddNode
		\end{align*}
		\LinkNodes[margin=4 cm]{\begin{minipage}{2cm}{Trasformo in moltiplicazione}
			\end{minipage}}
			\LinkNodes[margin=4 cm]{\begin{minipage}{2cm}
					Stesso indice moltiplico
				\end{minipage}}
				%				\LinkNodes[margin=4 cm]{\begin{minipage}{3cm}
				%						$mcd(12,19,10)=1$
				%					\end{minipage}}
				
			\end{NodesList}
		\end{exercise}
\begin{exercise}
	$\sqrt[3]{a}\div\sqrt[5]{b}$
	\tcblower
	La divisione è
	\begin{NodesList}
		\begin{align*}
		&\sqrt[3]{a}\div\sqrt[5]{b}\AddNode\\
		&\\
		&\csqrt{3}{a}\cdot\csqrt{5}{\dfrac{1}{b}}\AddNode\\
		&\\
		&\sqrt[15]{a^5}\cdot\sqrt[15]{\left(\dfrac{1}{b}\right)^3}\AddNode\\
		&\\
		&\sqrt[15]{\dfrac{a^5}{b^3}}\AddNode
		\end{align*}
		\LinkNodes[margin=4 cm]{\begin{minipage}{2cm}{Trasformo in moltiplicazione}
			\end{minipage}}
			\LinkNodes[margin=4 cm]{\begin{minipage}{2cm}
					Riduco allo stesso indice
				\end{minipage}}
								\LinkNodes[margin=4 cm]{\begin{minipage}{3cm}
										Moltiplico
									\end{minipage}}
				
			\end{NodesList}
		\end{exercise}
	\begin{exercise}
		$\sqrt[3]{\dfrac{a}{b}}\div\sqrt[5]{\dfrac{c}{d}}$
		\tcblower
		La divisione è
		\begin{NodesList}
			\begin{align*}
			&\csqrt{3}{\dfrac{a}{b}}\div\csqrt{5}{\dfrac{c}{d}}\AddNode\\
			&\\
			&\sqrt[3]{\dfrac{a}{b}}\cdot\sqrt[5]{\dfrac{d}{c}}\AddNode\\
			&\\
			&\sqrt[15]{\left( \dfrac{a}{b}\right)^5}\cdot\sqrt[15]{\left(\dfrac{c}{d}\right)^3 }\AddNode\\
			&\\
			&\sqrt[15]{\dfrac{a^5d^3}{b^5c^3}}\AddNode
			\end{align*}
			\LinkNodes[margin=4 cm]{\begin{minipage}{2cm}{Trasformo in moltiplicazione}
				\end{minipage}}
				\LinkNodes[margin=4 cm]{\begin{minipage}{2cm}
						Riduco allo stesso indice
					\end{minipage}}
					\LinkNodes[margin=4 cm]{\begin{minipage}{3cm}
							Moltiplico
						\end{minipage}}
						
					\end{NodesList}
				\end{exercise}
	\begin{exercise}
		$\sqrt[3]{2a^2}\div\sqrt[3]{\dfrac{4}{3}a^6b^3}$
		\tcblower
		La divisione è
		\begin{NodesList}
			\begin{align*}
			&\sqrt[3]{2a^2}\div\sqrt[3]{\dfrac{4}{3}a^6b^3}\AddNode\\
			&\\
			&\sqrt[3]{2a^2}\cdot\sqrt[3]{\dfrac{3}{4}\dfrac{1}{a^6b^3}}\AddNode\\
			&\\
			&\sqrt[3]{\dfrac{6}{4}\dfrac{a^2}{a^6b^3}}\AddNode\\
			&\\
			&\sqrt[3]{\dfrac{3}{2}\dfrac{1}{a^4b^3}}\AddNode
			\end{align*}
			\LinkNodes[margin=4 cm]{\begin{minipage}{2cm}{Trasformo in moltiplicazione}
				\end{minipage}}
				\LinkNodes[margin=4 cm]{\begin{minipage}{2cm}
						Stesso indice moltiplico
					\end{minipage}}
					\LinkNodes[margin=4 cm]{\begin{minipage}{3cm}
							Semplifico
						\end{minipage}}
						
					\end{NodesList}
				\end{exercise}
		\tcbstoprecording
			\newpage
			\section{Soluzioni divisioni}
			\tcbinputrecords
\section{Trasporto di un termine fuori dal segno di radice}
Senokifica i seguenti radicali
\tcbstartrecording
	\begin{exercise}
		$\sqrt{8}$
		\tcblower
		ottengo:
		\begin{NodesList}
			\begin{align*}
				&\sqrt{8}\AddNode\\
				&\\
				&=\sqrt{2^3}\AddNode\\
				&\\
				&=\sqrt{2^2}\cdot\sqrt{2}\AddNode\\
				&\\
				&=2\sqrt{2}\AddNode
			\end{align*}
			\LinkNodes[margin=4 cm]{\begin{minipage}{2cm}{Scompongo in fattori}
				\end{minipage}}
				\LinkNodes[margin=4 cm]{\begin{minipage}{2cm}
						$3=2+1$
					\end{minipage}}
					\LinkNodes[margin=4 cm]{\begin{minipage}{3cm}
							Semplifico
						\end{minipage}}
						
					\end{NodesList}
				\end{exercise}

	\begin{exercise}
		$\sqrt{8}$
		\tcblower
		ottengo:
		\begin{NodesList}
			\begin{align*}
			&\sqrt{8}\AddNode\\
			&\\
			&=\sqrt{2^3}\AddNode\\
			&\\
			&=2^{\circled{1}}\sqrt{2^{\rectangolo{1}}}\AddNode\\
%			&\\
%			&=2\sqrt{2}\AddNode
			\end{align*}
			\LinkNodes[margin=4 cm]{\begin{minipage}{2cm}{Scompongo in fattori}
				\end{minipage}}
				\LinkNodes[margin=4 cm]{\begin{minipage}{3cm}
					\opidiv{3}{2} 
					
					$  $$q=\circled{1}$ $r=\rectangolo{1}$
					\end{minipage}}
%					\LinkNodes[margin=4 cm]{\begin{minipage}{3cm}
%							Semplifico
%						\end{minipage}}
						
					\end{NodesList}
				\end{exercise}		
	\begin{exercise}
		$\sqrt[3]{112}$
		\tcblower
		ottengo:
		\begin{NodesList}
			\begin{align*}
			&\sqrt[3]{112}\AddNode\\
			&\\
			&=\sqrt[3]{2^4\cdot7}\AddNode\\
			&\\
			&=\sqrt[3]{2^3\cdot2\cdot7}\AddNode\\
			&\\
			&=2\sqrt[3]{2}\AddNode
			\end{align*}
			\LinkNodes[margin=4 cm]{\begin{minipage}{2cm}{Scompongo in fattori}
				\end{minipage}}
				\LinkNodes[margin=4 cm]{\begin{minipage}{2cm}
						$4=3+1$
					\end{minipage}}
					\LinkNodes[margin=4 cm]{\begin{minipage}{3cm}
							Semplifico
						\end{minipage}}
						
					\end{NodesList}
				\end{exercise}	
	\begin{exercise}
		$\sqrt[3]{112}$
		\tcblower
		ottengo:
		\begin{NodesList}
			\begin{align*}
			&\sqrt[3]{112}\AddNode\\
			&\\
			&=\sqrt[3]{2^4\cdot7}\AddNode\\
			&\\
			&=2^{\circled{1}}\sqrt[3]{2^{\rectangolo{1}}\cdot7}\AddNode\\
			&\\
			&=2\sqrt[3]{14}\AddNode
			\end{align*}
			\LinkNodes[margin=4 cm]{\begin{minipage}{2cm}{Scompongo in fattori}
				\end{minipage}}
				\LinkNodes[margin=4 cm]{\begin{minipage}{3cm}
						\opidiv{3}{2} 
						
						$  $$q=\circled{1}$ $r=\rectangolo{1}$
					\end{minipage}}
					\LinkNodes[margin=4 cm]{\begin{minipage}{3cm}
							Semplifico
						\end{minipage}}
								
					\end{NodesList}
				\end{exercise}										
				\tcbstoprecording
				\newpage
				\section{Soluzioni trasporto}
				\tcbinputrecords
	\input{Equazioni2grad}
	\backmatter
	\cleardoublepage
	\appendix
	\input{esempi}
}
\opt{terzo}{    %\input{equazioni_fraz}
	%\input{ncomplessi}
	\input{funzgonio}
	\input{trigonometriatab}
	\input{esempi}
	\backmatter
	\cleardoublepage
	\appendix
	\input{Tabelle_goniometriche}
}
\opt{quarto}{\input{disequazioni_primogrado}
	\input{disequazioni_secondogrado}
	\input{funzExpLog}
	\input{logaritmi}
	\backmatter
	\cleardoublepage
	\appendix
	\input{tabelle_disequazioni}
}
%\opt{extra}{\input{extra}}
%\opt{grafici}{\input{grafici}}
\input{../Mod_base/MezziUsati}
\addcontentsline{toc}{chapter}{\indexname}
\printindex
\end{document}

\chapter{Sistemi lineari}
\section{Metodo di sostituzione}
\label{sec:metosdosostituzione}
\label{cha:SistemiLineari}
\tcbstartrecording
\begin{exercise}
\[\begin{cases}
	x+2y=1\\
	2x+3y=3
\end{cases}
\]
	\tcblower
	Il prodotto è:
\begin{NodesList}
	\begin{align*}
	&\begin{cases}
	x+2y=1\AddNode\\
	2x+3y=3
	\end{cases}\\
	&\begin{cases}
	x=1-2y\AddNode\\
	2x+3y=3
	\end{cases}\\
	&\begin{cases}
	x=1-2y\\
	2(1-2y)+3y=3\AddNode\\
	\end{cases}\\
	&\begin{cases}
	x=1-2y\\
	2-4y+3y=3\AddNode\\
	\end{cases}\\
	&\begin{cases}
	y=-1\AddNode\\
	x=1-2y\\
	\end{cases}\\
	&\begin{cases}
	y=-1\\
	x=1-2(-1) \AddNode\\
	\end{cases}\\
	&\begin{cases}
	y=-1\\
	x=3 \AddNode\\
	\end{cases}\\
	\end{align*}
	\LinkNodes[margin=4 cm]{\begin{minipage}{2cm}{Risolvo rispetto alla x}
		\end{minipage}}
		\LinkNodes[margin=4 cm]{\begin{minipage}{2cm}
				Sostituisco la x nella seconda
			\end{minipage}}
			\LinkNodes[margin=4 cm]{\begin{minipage}{2cm}
					Semplifico
				\end{minipage}}
				\LinkNodes[margin=4 cm]{\begin{minipage}{2cm}
						Risolvo rispetto alla y
					\end{minipage}}
					\LinkNodes[margin=4 cm]{\begin{minipage}{2cm}
							Sostituisco la y nella seconda
						\end{minipage}}
						\LinkNodes[margin=4 cm]{\begin{minipage}{2cm}
								Risolvo rispetto alla x
							\end{minipage}}
						\end{NodesList}
			\end{exercise}
\tcbstoprecording
\newpage
\section{Soluzione metodo sostituzione}
\tcbinputrecords
\newpage
\section{Metodo del confronto}
\label{sec:metodoconfronto}
\tcbstartrecording
\begin{exercise}
	\[\begin{cases}
	x+2y=1\\
	2x+3y=3\\
	\end{cases}
	\]
	\tcblower
	Il prodotto è:
	\begin{NodesList}[dy=6pt]
		\begin{align*}
		&\begin{cases}
		x+2y=1\AddNode\\
		2x+3y=3\\
		\end{cases}\\
		&\\
		&\begin{cases}
		x=1-2y\AddNode\\
		x=\dfrac{3}{2}-\dfrac{3}{2}y\AddNode[2]\\
		\end{cases}\\
		&\\
		&\begin{cases}
		x=\dfrac{3}{2}-\dfrac{3}{2}y\AddNode[2]\AddNode[3]\\
		x=1-2y\AddNode[4]\\
		\end{cases}\\
		&\\
		&\begin{cases}
		1-2y=\dfrac{3}{2}-\dfrac{3}{2}y\AddNode[3]\AddNode[4]\\
		x=1-2y\\
		\end{cases}\\
		&\\
		&\begin{cases}
		2-4y=3-3y\AddNode[4]\AddNode[5]\\
		x=1-2y\\
		\end{cases}\\
		&\\
		&\begin{cases}
		3y-4y=3-2\AddNode[5]\AddNode[6]\\
		x=1-2y\\
		\end{cases}\\
		&\\
		&\begin{cases}
		y=-1\AddNode[6]\AddNode[7]\\
		x=1-2y\\
		\end{cases}\\
		&\\
		&\begin{cases}
		y=-1\\
		x=3\AddNode[7]\
		\end{cases}\\
		\end{align*}
		\LinkNodes[margin=4 cm]{\begin{minipage}{2cm}{Risolvo rispetto alla x}
			\end{minipage}}
			\LinkNodes[margin=1 cm]{\begin{minipage}{2cm}{Risolvo rispetto alla x}
				\end{minipage}}
				\LinkNodes[margin=.5cm]{\begin{minipage}{2cm}{Confronto le due soluzioni}
					\end{minipage}}
					\LinkNodes[margin=3cm]{\begin{minipage}{2cm}{Confronto le due soluzioni}
						\end{minipage}}
						\LinkNodes[margin=3cm]{\begin{minipage}{2cm}{Semplifico}
							\end{minipage}}
							\LinkNodes[margin=3cm]{\begin{minipage}{2cm}{Semplifico}
								\end{minipage}}
								\LinkNodes[margin=3cm]{\begin{minipage}{2cm}{Risolvo rispetto a y}\end{minipage}}
								\LinkNodes[margin=3cm]{\begin{minipage}{2cm}{Sostituisco e risolvo rispetto a x}\end{minipage}}
							\end{NodesList}
						\end{exercise}
						\tcbstoprecording
						\newpage
						\section{Soluzione metodo confronto}
						\tcbinputrecords

						\newpage
\section{Metodo di somma sottrazione}
\label{sec:metododommasottrazione}
\tcbstartrecording
\begin{exercise}
	\[\begin{cases}
	x+2y=1\\
	2x+3y=3
	\end{cases}
	\]
	\tcblower
	Il prodotto è:
	\begin{NodesList}%[dy=6pt]
		\begin{align*}
		&\begin{cases}
		x+2y=1\AddNode\\
		2x+3y=3
		\end{cases}\\
		&\\
		&\begin{cases}
		\tikzmark{1}2x+4y=2\AddNode\\
		2x+3y=3
		%\tikz[remember picture, overlay]{\node[below=0.1cm and 0cm of 1](3){};}
		%\tikz[remember picture, overlay]{\node[below=0.1cm and 0cm of 2](4){};}
		\tikz[remember picture, overlay]{\node[left=2cm and 0cm of 1](5){$2$};}
		%\tikz[remember picture, overlay]{\node[below right=0.2cm and 0cm of 5](6) {$0-y=-13$};}
		%\tikz[remember picture, overlay]{\path[draw] (3) edge  (4);}
		\end{cases}\\
		&\\
		&\begin{cases}
		\tikzmark{1}2x+4y=2\\%\AddNode\\
		\tikzmark{0}2x+3y=3\tikzmark{2}
		\tikz[remember picture, overlay]{\node[below=0.2cm and 0cm of 0](3){};}
		\tikz[remember picture, overlay]{\node[below=0.2cm and 0cm of 2](4){};}
		\tikz[remember picture, overlay]{\node[left=2cm and 0cm of 1](5){$2$};}
		\tikz[remember picture, overlay]{\node[below right=0.4cm and -0.1cm of 0](6) {$0+y=-1$};}\AddNode\\
		\tikz[remember picture, overlay]{\path[draw] (3) edge  (4);}
		\end{cases}\\
		\intertext{Riprendo il sistema iniziale}
		&\begin{cases}
		x+2y=1\AddNode[2]\\
		2x+3y=3
		\end{cases}\\
		&\\
		&\begin{cases}
		\tikzmark{1}3x+6y=3\\
		\tikzmark{6}4x+6y=6\AddNode[2]
		%\tikz[remember picture, overlay]{\node[below=0.1cm and 0cm of 1](3){};}
		%\tikz[remember picture, overlay]{\node[below=0.1cm and 0cm of 2](4){};}
		\tikz[remember picture, overlay]{\node[left=2cm and 0cm of 1](5){$3$};}
		\tikz[remember picture, overlay]{\node[left=2cm and 0cm of 6](7){$2$};}
		%\tikz[remember picture, overlay]{\node[below right=0.2cm and 0cm of 5](6) {$0-y=-13$};}
		%\tikz[remember picture, overlay]{\path[draw] (3) edge  (4);}
		\end{cases}\\
		&\\
		&\begin{cases}
		\tikzmark{1}3x+6y=3\\
		\tikzmark{6}\tikzmark{0}4x+6y=6\tikzmark{2}
		\tikz[remember picture, overlay]{\node[below=0.2cm and 0cm of 0](3){};}
		\tikz[remember picture, overlay]{\node[below=0.2cm and 0cm of 2](4){};}
		\tikz[remember picture, overlay]{\node[left=2cm and 0cm of 1](5){$3$};}
		\tikz[remember picture, overlay]{\node[left=2cm and 0cm of 6](7){$2$};}
		\tikz[remember picture, overlay]{\node[below right=0.4cm and -0.1cm of 0](6) {$-x+0=-3$};}\AddNode[2]\\
		\tikz[remember picture, overlay]{\path[draw] (3) edge  (4);}
		\end{cases}\\
		\end{align*}
		\LinkNodes[margin=2 cm]{\begin{minipage}{2cm}{Moltiplico la riga per 2}
			\end{minipage}}
			\LinkNodes[margin=2 cm]{\begin{minipage}{2cm}{Sottraggo la seconda riga alla prima e ottengo x}
				\end{minipage}}
				%	\LinkNodes[margin=4 cm]{\begin{minipage}{2cm}{Ottengo y}
				%		\end{minipage}}
				\LinkNodes{\begin{minipage}{2cm}{Moltiplico la prima riga per 3, la seconda per 2}
					\end{minipage}}
					\LinkNodes[margin=2 cm]{\begin{minipage}{2cm}{Sottraggo la seconda riga alla prima e ottengo y}
						\end{minipage}}
						%	\LinkNodes[margin=4 cm]{\begin{minipage}{2cm}{Ottengo x}
						%		\end{minipage}}
					\end{NodesList}
						\end{exercise}
						\tcbstoprecording
						\newpage
						\section{Soluzione somma e sottrazione}
						\tcbinputrecords


\chapter{Radicali}
\section{Prodotti}
Calcola il prodotto fra questi radicali:
\tcbstartrecording
\begin{exercise}
$\sqrt[3]{a^4b}\cdot\sqrt[4]{ab^2}$
	\tcblower
	Il prodotto è:
\begin{NodesList}
	\begin{align*}
		&\csqrt{3}{a^4b}\cdot\csqrt{4}{ab^2}\AddNode\\ &\\
		&\sqrt[12]{a^{16}b^4}\cdot\sqrt[12]{a^3b^6}\AddNode\\ &\\
		&\sqrt[12]{a^{19}b^{10}}\AddNode\\ &\\
		&\sqrt[12]{a^{19}b^{10}}\AddNode\\ &\\
	\end{align*}
	\LinkNodes[margin=4 cm]{\begin{minipage}{2cm}{Riduco allo stesso indice}
		\end{minipage}}
		\LinkNodes[margin=4 cm]{\begin{minipage}{2cm}
				Moltiplico
			\end{minipage}}
			\LinkNodes[margin=4 cm]{\begin{minipage}{3cm}
					$mcd(12,19,10)=1$
				\end{minipage}}
			
						\end{NodesList}
\end{exercise}
\begin{exercise}
	$\sqrt[8]{\dfrac{2a^3(a+b)}{3}}\cdot\sqrt[4]{\dfrac{3}{4}\dfrac{a}{b(a+b)}}$
	\tcblower
	Il prodotto è:
	\begin{NodesList}
		\begin{align*}
			&\csqrt{8}{\dfrac{2a^3(a+b)}{3}}\cdot\csqrt{4}{\dfrac{3}{4}\dfrac{a}{b(a+b)}}\AddNode\\
			&\\
			&\sqrt[8]{\dfrac{2a^3(a+b)}{3}}\cdot\sqrt[8]{\dfrac{9}{16}\dfrac{a^2}{b^2(a+b)^2}}\AddNode\\
			&\\
			&\sqrt[8]{\dfrac{18}{16}\dfrac{a^5(a+b)}{b^2(a+b)^2}}\AddNode\\
			&\\
			&\sqrt[8]{\dfrac{9}{8}\dfrac{a^5}{b^2(a+b)}}\AddNode\\
		\end{align*}
		\LinkNodes[margin=4 cm]{\begin{minipage}{2cm}{Riduco allo stesso indice}
			\end{minipage}}
			\LinkNodes[margin=4 cm]{\begin{minipage}{2cm}
					Moltiplico
				\end{minipage}}
				\LinkNodes[margin=4 cm]{\begin{minipage}{3cm}
						Semplifico
					\end{minipage}}
					
				\end{NodesList}
			\end{exercise}
			
\begin{exercise}
	$\sqrt[12]{ab^3}\cdot\sqrt[15]{\dfrac{1}{ab(a+b)}}$
	\tcblower
	Il prodotto è:
	\begin{NodesList}
		\begin{align*}
		&\csqrt{12}{ab^3}\cdot\csqrt{15}{\dfrac{1}{ab(a+b)}}\AddNode\\
		&\\
		&\sqrt[60]{a^5b^{15}}\cdot\sqrt[60]{\dfrac{1}{a^4b^4(a+b)^4}}\AddNode\\
		&\\
		&\sqrt[60]{\dfrac{a^5b^{15}}{a^4b^4(a+b)^4}}\AddNode\\
		&\\
		&\sqrt[60]{\dfrac{ab^{9}}{(a+b)^4}}\AddNode\\
		\end{align*}
		\LinkNodes[margin=4 cm]{\begin{minipage}{2cm}{Riduco allo stesso indice}
			\end{minipage}}
			\LinkNodes[margin=4 cm]{\begin{minipage}{2cm}
					Moltiplico
				\end{minipage}}
				\LinkNodes[margin=4 cm]{\begin{minipage}{3cm}
						Semplifico
					\end{minipage}}
					
				\end{NodesList}
			\end{exercise}
\begin{exercise}
	$\sqrt{c}\cdot\sqrt[3]{a+b}$
	\tcblower
	Il prodotto è:
	\begin{NodesList}
		\begin{align*}
		&\sqrt{c}\cdot\csqrt{3}{a+b}\AddNode\\
		&\\
		&\sqrt[6]{c^3}\cdot\sqrt[6]{(a+b)^2}\AddNode\\
		&\\
		&\sqrt[6]{c^3(a+b)^2}\AddNode\\
		&\\
%		&\sqrt[60]{\dfrac{ab^{9}}{(a+b)^4}}\AddNode\\
		\end{align*}
		\LinkNodes[margin=4 cm]{\begin{minipage}{2cm}{Riduco allo stesso indice}
			\end{minipage}}
			\LinkNodes[margin=4 cm]{\begin{minipage}{2cm}
					Moltiplico
				\end{minipage}}
%				\LinkNodes[margin=4 cm]{\begin{minipage}{3cm}
%						Semplifico
%					\end{minipage}}
					
				\end{NodesList}
			\end{exercise}		
\begin{exercise}
	$\sqrt{c}\cdot\sqrt[2]{a+b}$
	\tcblower
	Il prodotto è:
	\begin{NodesList}
		\begin{align*}
		&\sqrt{c}\cdot\sqrt[2]{a+b}\AddNode\\
		&\\
		&\sqrt{c(a+b)}\AddNode\\
		&\\
		&\sqrt[2]{ac+bc}\AddNode\\
		%		&\sqrt[60]{\dfrac{ab^{9}}{(a+b)^4}}\AddNode\\
		\end{align*}
		\LinkNodes[margin=4 cm]{\begin{minipage}{2cm}{Moltiplico}
			\end{minipage}}
			\LinkNodes[margin=4 cm]{\begin{minipage}{2cm}
					Semplifico a piacere
				\end{minipage}}
				%				\LinkNodes[margin=4 cm]{\begin{minipage}{3cm}
				%						Semplifico
				%					\end{minipage}}
				
			\end{NodesList}
		\end{exercise}					
%\begin{exercise}[no solution]
%	It holds:
%	\begin{equation*}
%	\frac{d}{dx}\left(\ln|x|\right) = \frac{1}{x}.
%	\end{equation*}
%\end{exercise}

\tcbstoprecording
\newpage
\section{Soluzioni Prodotti}
\tcbinputrecords
\newpage
\section{Divisioni}
Calcola il quoziente fra questi radicali:
\tcbstartrecording
\begin{exercise}
	$\sqrt[3]{a}\div\sqrt[3]{b}$
	\tcblower
	La divisione è
	\begin{NodesList}
		\begin{align*}
		&\sqrt{3}{a}\div\sqrt{3}{b}\AddNode\\
		&\\
		&\sqrt[3]{a}\cdot\sqrt[3]{\dfrac{1}{b}}\AddNode\\
		&\\
		&\sqrt[3]{\dfrac{a}{b}}\AddNode
		\end{align*}
		\LinkNodes[margin=4 cm]{\begin{minipage}{2cm}{Trasformo in moltiplicazione}
			\end{minipage}}
			\LinkNodes[margin=4 cm]{\begin{minipage}{2cm}
					Stesso indice moltiplico
				\end{minipage}}
%				\LinkNodes[margin=4 cm]{\begin{minipage}{3cm}
%						$mcd(12,19,10)=1$
%					\end{minipage}}
					
				\end{NodesList}
			\end{exercise}
			
\begin{exercise}
	$\sqrt[3]{\dfrac{a}{b}}\div\sqrt[3]{\dfrac{c}{d}}$
	\tcblower
	La divisione è
	\begin{NodesList}
		\begin{align*}
		&\sqrt[3]{\dfrac{a}{b}}\div\sqrt[3]{\dfrac{c}{d}}\AddNode\\
		&\\
		&\sqrt[3]{\dfrac{a}{b}}\cdot\sqrt[3]{\dfrac{d}{c}}\AddNode\\
		&\\
		&\sqrt[3]{\dfrac{a}{b}\cdot\dfrac{d}{c}}\AddNode
		\end{align*}
		\LinkNodes[margin=4 cm]{\begin{minipage}{2cm}{Trasformo in moltiplicazione}
			\end{minipage}}
			\LinkNodes[margin=4 cm]{\begin{minipage}{2cm}
					Stesso indice moltiplico
				\end{minipage}}
				%				\LinkNodes[margin=4 cm]{\begin{minipage}{3cm}
				%						$mcd(12,19,10)=1$
				%					\end{minipage}}
				
			\end{NodesList}
		\end{exercise}
\begin{exercise}
	$\sqrt[3]{a}\div\sqrt[5]{b}$
	\tcblower
	La divisione è
	\begin{NodesList}
		\begin{align*}
		&\sqrt[3]{a}\div\sqrt[5]{b}\AddNode\\
		&\\
		&\csqrt{3}{a}\cdot\csqrt{5}{\dfrac{1}{b}}\AddNode\\
		&\\
		&\sqrt[15]{a^5}\cdot\sqrt[15]{\left(\dfrac{1}{b}\right)^3}\AddNode\\
		&\\
		&\sqrt[15]{\dfrac{a^5}{b^3}}\AddNode
		\end{align*}
		\LinkNodes[margin=4 cm]{\begin{minipage}{2cm}{Trasformo in moltiplicazione}
			\end{minipage}}
			\LinkNodes[margin=4 cm]{\begin{minipage}{2cm}
					Riduco allo stesso indice
				\end{minipage}}
								\LinkNodes[margin=4 cm]{\begin{minipage}{3cm}
										Moltiplico
									\end{minipage}}
				
			\end{NodesList}
		\end{exercise}
	\begin{exercise}
		$\sqrt[3]{\dfrac{a}{b}}\div\sqrt[5]{\dfrac{c}{d}}$
		\tcblower
		La divisione è
		\begin{NodesList}
			\begin{align*}
			&\csqrt{3}{\dfrac{a}{b}}\div\csqrt{5}{\dfrac{c}{d}}\AddNode\\
			&\\
			&\sqrt[3]{\dfrac{a}{b}}\cdot\sqrt[5]{\dfrac{d}{c}}\AddNode\\
			&\\
			&\sqrt[15]{\left( \dfrac{a}{b}\right)^5}\cdot\sqrt[15]{\left(\dfrac{c}{d}\right)^3 }\AddNode\\
			&\\
			&\sqrt[15]{\dfrac{a^5d^3}{b^5c^3}}\AddNode
			\end{align*}
			\LinkNodes[margin=4 cm]{\begin{minipage}{2cm}{Trasformo in moltiplicazione}
				\end{minipage}}
				\LinkNodes[margin=4 cm]{\begin{minipage}{2cm}
						Riduco allo stesso indice
					\end{minipage}}
					\LinkNodes[margin=4 cm]{\begin{minipage}{3cm}
							Moltiplico
						\end{minipage}}
						
					\end{NodesList}
				\end{exercise}
	\begin{exercise}
		$\sqrt[3]{2a^2}\div\sqrt[3]{\dfrac{4}{3}a^6b^3}$
		\tcblower
		La divisione è
		\begin{NodesList}
			\begin{align*}
			&\sqrt[3]{2a^2}\div\sqrt[3]{\dfrac{4}{3}a^6b^3}\AddNode\\
			&\\
			&\sqrt[3]{2a^2}\cdot\sqrt[3]{\dfrac{3}{4}\dfrac{1}{a^6b^3}}\AddNode\\
			&\\
			&\sqrt[3]{\dfrac{6}{4}\dfrac{a^2}{a^6b^3}}\AddNode\\
			&\\
			&\sqrt[3]{\dfrac{3}{2}\dfrac{1}{a^4b^3}}\AddNode
			\end{align*}
			\LinkNodes[margin=4 cm]{\begin{minipage}{2cm}{Trasformo in moltiplicazione}
				\end{minipage}}
				\LinkNodes[margin=4 cm]{\begin{minipage}{2cm}
						Stesso indice moltiplico
					\end{minipage}}
					\LinkNodes[margin=4 cm]{\begin{minipage}{3cm}
							Semplifico
						\end{minipage}}
						
					\end{NodesList}
				\end{exercise}
		\tcbstoprecording
			\newpage
			\section{Soluzioni divisioni}
			\tcbinputrecords
\section{Trasporto di un termine fuori dal segno di radice}
Senokifica i seguenti radicali
\tcbstartrecording
	\begin{exercise}
		$\sqrt{8}$
		\tcblower
		ottengo:
		\begin{NodesList}
			\begin{align*}
				&\sqrt{8}\AddNode\\
				&\\
				&=\sqrt{2^3}\AddNode\\
				&\\
				&=\sqrt{2^2}\cdot\sqrt{2}\AddNode\\
				&\\
				&=2\sqrt{2}\AddNode
			\end{align*}
			\LinkNodes[margin=4 cm]{\begin{minipage}{2cm}{Scompongo in fattori}
				\end{minipage}}
				\LinkNodes[margin=4 cm]{\begin{minipage}{2cm}
						$3=2+1$
					\end{minipage}}
					\LinkNodes[margin=4 cm]{\begin{minipage}{3cm}
							Semplifico
						\end{minipage}}
						
					\end{NodesList}
				\end{exercise}

	\begin{exercise}
		$\sqrt{8}$
		\tcblower
		ottengo:
		\begin{NodesList}
			\begin{align*}
			&\sqrt{8}\AddNode\\
			&\\
			&=\sqrt{2^3}\AddNode\\
			&\\
			&=2^{\circled{1}}\sqrt{2^{\rectangolo{1}}}\AddNode\\
%			&\\
%			&=2\sqrt{2}\AddNode
			\end{align*}
			\LinkNodes[margin=4 cm]{\begin{minipage}{2cm}{Scompongo in fattori}
				\end{minipage}}
				\LinkNodes[margin=4 cm]{\begin{minipage}{3cm}
					\opidiv{3}{2} 
					
					$  $$q=\circled{1}$ $r=\rectangolo{1}$
					\end{minipage}}
%					\LinkNodes[margin=4 cm]{\begin{minipage}{3cm}
%							Semplifico
%						\end{minipage}}
						
					\end{NodesList}
				\end{exercise}		
	\begin{exercise}
		$\sqrt[3]{112}$
		\tcblower
		ottengo:
		\begin{NodesList}
			\begin{align*}
			&\sqrt[3]{112}\AddNode\\
			&\\
			&=\sqrt[3]{2^4\cdot7}\AddNode\\
			&\\
			&=\sqrt[3]{2^3\cdot2\cdot7}\AddNode\\
			&\\
			&=2\sqrt[3]{2}\AddNode
			\end{align*}
			\LinkNodes[margin=4 cm]{\begin{minipage}{2cm}{Scompongo in fattori}
				\end{minipage}}
				\LinkNodes[margin=4 cm]{\begin{minipage}{2cm}
						$4=3+1$
					\end{minipage}}
					\LinkNodes[margin=4 cm]{\begin{minipage}{3cm}
							Semplifico
						\end{minipage}}
						
					\end{NodesList}
				\end{exercise}	
	\begin{exercise}
		$\sqrt[3]{112}$
		\tcblower
		ottengo:
		\begin{NodesList}
			\begin{align*}
			&\sqrt[3]{112}\AddNode\\
			&\\
			&=\sqrt[3]{2^4\cdot7}\AddNode\\
			&\\
			&=2^{\circled{1}}\sqrt[3]{2^{\rectangolo{1}}\cdot7}\AddNode\\
			&\\
			&=2\sqrt[3]{14}\AddNode
			\end{align*}
			\LinkNodes[margin=4 cm]{\begin{minipage}{2cm}{Scompongo in fattori}
				\end{minipage}}
				\LinkNodes[margin=4 cm]{\begin{minipage}{3cm}
						\opidiv{3}{2} 
						
						$  $$q=\circled{1}$ $r=\rectangolo{1}$
					\end{minipage}}
					\LinkNodes[margin=4 cm]{\begin{minipage}{3cm}
							Semplifico
						\end{minipage}}
								
					\end{NodesList}
				\end{exercise}										
				\tcbstoprecording
				\newpage
				\section{Soluzioni trasporto}
				\tcbinputrecords
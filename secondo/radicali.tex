\chapter{Radicali}
\section{Prodotti}
\tcbstartrecording

\begin{exercise}
	Calcola il prodotto fra questi radicali:
%	\begin{equation*}
	$\sqrt[3]{a^4b}\cdot\sqrt[4]{ab^2}$
%	\end{equation*}
	\tcblower
	Il prodotto è:
\begin{NodesList}
	\begin{align*}
		&\sqrt[3]{a^4b}\cdot\sqrt[4]{ab^2}\AddNode\\
		&\sqrt[12]{a^{16}b^4}\cdot\sqrt[12]{a^3b^6}\AddNode\\
		&\sqrt[12]{a^{19}b^{10}}\AddNode\\
		&\sqrt[12]{a^{19}b^{10}}\AddNode\\
	\end{align*}
	\LinkNodes[margin=4 cm]{\begin{minipage}{2cm}{Riduco allo stesso indice}
		\end{minipage}}
		\LinkNodes[margin=4 cm]{\begin{minipage}{2cm}
				Moltiplico
			\end{minipage}}
			\LinkNodes[margin=4 cm]{\begin{minipage}{3cm}
					$mcd(12,19,10)=1$
				\end{minipage}}
			
						\end{NodesList}
\end{exercise}

%\begin{exercise}[no solution]
%	It holds:
%	\begin{equation*}
%	\frac{d}{dx}\left(\ln|x|\right) = \frac{1}{x}.
%	\end{equation*}
%\end{exercise}

\tcbstoprecording
\newpage
\section{Soluzioni Prodotti}
\tcbinputrecords
\newpage
\section{Divisioni}
\tcbstartrecording
\begin{exercise}
	Calcola il quoziente fra questi radicali:
	\begin{equation*}
	\sqrt[3]{a}\div\sqrt[3]{b}
	\end{equation*}
	\tcblower
	Il divisione è:
	\begin{NodesList}
		\begin{align*}
		&\sqrt[3]{a}\div\sqrt[3]{b}\AddNode\\
		&\sqrt[3]{a}\cdot\sqrt[3]{\dfrac{1}{b}}\AddNode\\
		&\sqrt[3]{\dfrac{a}{b}}\AddNode
		\end{align*}
		\LinkNodes[margin=4 cm]{\begin{minipage}{2cm}{Trasformo in moltiplicazione}
			\end{minipage}}
			\LinkNodes[margin=4 cm]{\begin{minipage}{2cm}
					Stesso indice moltiplico
				\end{minipage}}
%				\LinkNodes[margin=4 cm]{\begin{minipage}{3cm}
%						$mcd(12,19,10)=1$
%					\end{minipage}}
					
				\end{NodesList}
			\end{exercise}
			
\begin{exercise}
	Calcola il quoziente fra questi radicali:
	\begin{equation*}
	\sqrt[3]{\dfrac{a}{b}}\div\sqrt[3]{\dfrac{c}{d}}
	\end{equation*}
	\tcblower
	Il divisione è:
	\begin{NodesList}
		\begin{align*}
		&\sqrt[3]{\dfrac{a}{b}}\div\sqrt[3]{\dfrac{c}{d}}\AddNode\\
		&\sqrt[3]{\dfrac{a}{b}}\cdot\sqrt[3]{\dfrac{d}{c}}\AddNode\\
		&\sqrt[3]{\dfrac{a}{b}\cdot\dfrac{d}{c}}\AddNode
		\end{align*}
		\LinkNodes[margin=4 cm]{\begin{minipage}{2cm}{Trasformo in moltiplicazione}
			\end{minipage}}
			\LinkNodes[margin=4 cm]{\begin{minipage}{2cm}
					Stesso indice moltiplico
				\end{minipage}}
				%				\LinkNodes[margin=4 cm]{\begin{minipage}{3cm}
				%						$mcd(12,19,10)=1$
				%					\end{minipage}}
				
			\end{NodesList}
		\end{exercise}
\begin{exercise}
	Calcola il quoziente fra questi radicali:
	\begin{equation*}
	\sqrt[3]{a}\div\sqrt[5]{b}
	\end{equation*}
	\tcblower
	Il divisione è:
	\begin{NodesList}
		\begin{align*}
		&\sqrt[3]{a}\div\sqrt[5]{b}\AddNode\\
		&\sqrt[3]{a}\cdot\sqrt[5]{\dfrac{1}{b}}\AddNode\\
		&\sqrt[15]{a^5}\cdot\sqrt[15]{\left(\dfrac{1}{b}\right)^3}\AddNode\\
		&\sqrt[15]{\dfrac{a^5}{b^3}}\AddNode
		\end{align*}
		\LinkNodes[margin=4 cm]{\begin{minipage}{2cm}{Trasformo in moltiplicazione}
			\end{minipage}}
			\LinkNodes[margin=4 cm]{\begin{minipage}{2cm}
					Riduco allo stesso indice
				\end{minipage}}
								\LinkNodes[margin=4 cm]{\begin{minipage}{3cm}
										Moltiplico
									\end{minipage}}
				
			\end{NodesList}
		\end{exercise}
	\begin{exercise}
		Calcola il quoziente fra questi radicali:
		\begin{equation*}
		\sqrt[3]{\dfrac{a}{b}}\div\sqrt[5]{\dfrac{c}{d}}
		\end{equation*}
		\tcblower
		Il divisione è:
		\begin{NodesList}
			\begin{align*}
			&\sqrt[3]{\dfrac{a}{b}}\div\sqrt[5]{\dfrac{c}{d}}\AddNode\\
			&\sqrt[3]{\dfrac{a}{b}}\cdot\sqrt[5]{\dfrac{d}{c}}\AddNode\\
			&\sqrt[15]{\left( \dfrac{a}{b}\right)^5}\cdot\sqrt[15]{\left(\dfrac{c}{d}\right)^3 }\AddNode\\
			&\sqrt[15]{\dfrac{a^5d^3}{b^5c^3}}\AddNode
			\end{align*}
			\LinkNodes[margin=4 cm]{\begin{minipage}{2cm}{Trasformo in moltiplicazione}
				\end{minipage}}
				\LinkNodes[margin=4 cm]{\begin{minipage}{2cm}
						Riduco allo stesso indice
					\end{minipage}}
					\LinkNodes[margin=4 cm]{\begin{minipage}{3cm}
							Moltiplico
						\end{minipage}}
						
					\end{NodesList}
				\end{exercise}
	\begin{exercise}
		Calcola il quoziente fra questi radicali:
		\begin{equation*}
		\sqrt[3]{2a^2}\div\sqrt[3]{\dfrac{4}{3}a^6b^3}
		\end{equation*}
		\tcblower
		Il divisione è:
		\begin{NodesList}
			\begin{align*}
			&\sqrt[3]{2a^2}\div\sqrt[3]{\dfrac{4}{3}a^6b^3}\AddNode\\
			&\sqrt[3]{2a^2}\cdot\sqrt[3]{\dfrac{3}{4}\dfrac{1}{a^6b^3}}\AddNode\\
			&\sqrt[3]{\dfrac{6}{4}\dfrac{a^2}{a^6b^3}}\AddNode\\
			&\sqrt[3]{\dfrac{3}{2}\dfrac{1}{a^4b^3}}\AddNode
			\end{align*}
			\LinkNodes[margin=4 cm]{\begin{minipage}{2cm}{Trasformo in moltiplicazione}
				\end{minipage}}
				\LinkNodes[margin=4 cm]{\begin{minipage}{2cm}
						Stesso indice moltiplico
					\end{minipage}}
					\LinkNodes[margin=4 cm]{\begin{minipage}{3cm}
							Semplifico
						\end{minipage}}
						
					\end{NodesList}
				\end{exercise}
		\tcbstoprecording
			\newpage
			\section{Soluzioni divisioni}
			\tcbinputrecords
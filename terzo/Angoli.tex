\chapter{Angoli}
\label{cha:angolibase}
%\num[round-precision=2,round-mode=places]{1.678}
\section{Da grado sessagesimale a sessadecimale}
\begin{esempiot}{}{}
Trasformare $\alpha=\ang{52;38;28}$ in forma sessa-decimale\index{Grado!Sessadecimale}\index{Grado!Sessagesimale}
\end{esempiot}
\[\alpha=\ang{52}+\left(\dfrac{38}{60}\right)^{\degree }+\left(\dfrac{28}{3600}\right)^{\degree }=\ang{52}+\left(\dfrac{38\cdot60+28}{3600}\right)^{\degree } =\ang{52}+\left(\dfrac{2308}{3600}\right)^{\degree }\approx\ang[round-precision=2,round-mode=places]{52,641111}\]
\stampapuntini
\begin{esempiot}{}{}
	Trasformare $\alpha=\ang{75;55;35}$ in forma sessa-decimale\index{Grado!Sessadecimale}\index{Grado!Sessagesimale}
\end{esempiot}
\[\alpha=\ang{75}+\left(\dfrac{\puntini{55}}{60}\right)^{\degree }+\left(\dfrac{\puntini{35}}{3600}\right)^{\degree }=\ang{75}+\left(\dfrac{\puntini{55\cdot60+35}}{3600}\right)^{\degree } =\ang{75}+\left(\dfrac{\puntini{3335}}{3600}\right)^{\degree }\approx\puntini{\ang[round-precision=2,round-mode=places]{75,92638}}\]
\nonstampapuntini
\section{Da grado sessadecimale a sessagesimale}
\begin{esempiot}{}{}
	Convertire $\alpha=\ang{75.84}$
\end{esempiot}
Iniziamo con\index{Grado!Sessadecimale}\index{Grado!Sessagesimale}
$\alpha=\ang{75}+\ang{0.84}$ in gradi sessagesimali
\begin{align*}
\alpha^{\degree}&=\ang{75}\\ 
\alpha^{\arcminute}&=\ang{0.84}\cdot 60=\ang{;50.4;}=\ang{;50;}\\
\alpha^{\arcsecond}&=\ang{;0.4;}\cdot 60=\ang{;;24}\\
\end{align*}
abbiamo quindi
\[\alpha=\ang{75.84}=\ang{75;50;24}\]
\stampapuntini
\begin{esempiot}{}{}
	Convertire $\alpha=\ang{45.35}$ in gradi sessagesimali
\end{esempiot}
Iniziamo con 
$\alpha=\ang{45}+\ang{0.35}$
\begin{align*}
\alpha^{\degree}&=\puntini{\ang{45}}\\ 
\alpha^{\arcminute}&=\puntini{\ang{0.35}}\cdot 60=\ang{;21;}\\
\alpha^{\arcsecond}&=\puntini{\ang{;0.0;}}\cdot 60=\ang{;;0}\\
\end{align*}
abbiamo quindi
\[\alpha=\ang{45.35}=\ang{45;21;0}\]
\nonstampapuntini
\section{Conversioni radianti gradi}
\begin{esempiot}{}{}
Convertire $\alpha=\ang{45;58;25}$ in radianti\index{Radianti}
\end{esempiot}
Prima convertiamo in gradi \index{Radiante}\index{Grado!Sessagesimale}
\[\alpha=\ang{45}+\left(\dfrac{58}{60}\right)^{\degree }+\left(\dfrac{25}{3600}\right)^{\degree }=\ang{45}+\left(\dfrac{58\cdot60+25}{3600}\right)^{\degree } =\ang{45}+\left(\dfrac{3505}{3600}\right)^{\degree }\approx\ang[round-precision=4,round-mode=places]{45.97361111}\]
\[\rho=\dfrac{\pi}{180}\alpha\approx\dfrac{\pi}{180}\cdot\ang[round-precision=4,round-mode=places]{45.97361111}\approx\SI[round-precision=2,round-mode=places]{0.802390882}{\radian}\]
\stampapuntini
\begin{esempiot}{}{}
	Convertire $\alpha=\ang{70;48;25}$ in radianti
\end{esempiot}
Prima convertiamo in gradi \index{Radiante}\index{Grado!Sessagesimale}\index{Grado!Sessadecimale}
\[\alpha=\puntini{\ang{70}}+\left(\dfrac{48}{60}\right)^{\degree }+\left(\dfrac{25}{3600}\right)^{\degree }=\ang{70}+\left(\dfrac{\puntini{48\cdot60}+25}{3600}\right)^{\degree } =\ang{70}+\left(\dfrac{\puntini{2905}}{3600}\right)^{\degree }\approx\ang[round-precision=4,round-mode=places]{70.806944}\]
\[\rho=\dfrac{\puntini{\pi}}{180}\alpha\approx\dfrac{\puntini{\pi}}{180}\cdot\ang[round-precision=4,round-mode=places]{70.806944}\approx\puntini{\SI[round-precision=2,round-mode=places]{1.2358143}{\radian}}\]
\nonstampapuntini
\begin{esempiot}{}{}
	Convertire\SI[round-precision=3,round-mode=places]{2.856}{\radian} in gradi sessagesimali
\end{esempiot}
\[\alpha=\dfrac{180}{\pi}\cdot\rho=\dfrac{180}{\pi}\cdot\SI[round-precision=3,round-mode=places]{2.856}{\radian}\approx\ang[round-precision=4,round-mode=places]{163.6367463}\]
Iniziamo con 
$\alpha=\ang{163}+\ang[round-precision=4,round-mode=places]{0.6367463}$
\begin{align*}
\alpha^{\degree }&=\ang{163}\\ 
\alpha^{\arcminute}&=\ang[round-precision=4,round-mode=places]{0.6367463;;}\cdot 60=\ang[round-precision=4,round-mode=places]{;38.20477736;}=\ang{;38;}\\
\alpha^{\arcsecond}&=\ang[round-precision=4,round-mode=places]{;0.204777358;}\cdot 60\approx\ang{;;12}\\
\end{align*}
abbiamo quindi
\[\alpha=\ang[round-precision=4,round-mode=places]{163.6367463}=\ang{163;38;12}\]
\stampapuntini
\begin{esempiot}{}{}
	Convertire\SI[round-precision=3,round-mode=places]{0.823310}{\radian} in gradi sessagesimali
\end{esempiot}
\[\alpha=\dfrac{180}{\pi}\cdot\rho=\dfrac{180}{\pi}\cdot\SI[round-precision=3,round-mode=places]{0.823310}{\radian}\approx\puntini{\ang[round-precision=4,round-mode=places]{47.17218823}}\]
Iniziamo con 
$\alpha=\ang{47}+\ang[round-precision=4,round-mode=places]{0.17218823}$
\begin{align*}
\alpha^{\degree}&=\puntini{\ang{47}}\\ 
\alpha^{\arcminute}&=\ang[round-precision=4,round-mode=places]{0.17218823;;}\cdot 60=\puntini{\ang[round-precision=4,round-mode=places]{;10.33129385;}}=\puntini{\ang{;10;}}\\
\alpha^{\arcsecond}&=\puntini{\ang[round-precision=4,round-mode=places]{;0.331293854;}\cdot 60\approx\ang{;;19}}\\
\end{align*}
abbiamo quindi
\[\alpha=\ang[round-precision=4,round-mode=places]{47.17218823}=\ang{47;10;19}\]
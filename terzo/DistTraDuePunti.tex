\chapter{Distanza tra due punti}
\label{cha:DistanzaTraduePunti}
\begin{esempiot}{Distanza con stessa ordinata}{disuno}
Dati i punti $A\coord{1}{2}$\,$B\coord{3}{2}$ calcolare la distanza tra $A$ e $B$
\end{esempiot}
I due punti hanno la stessa ordinata quindi:
\begin{equation*}
d(AB)=\abs{x_1-x_2}=\abs{1-3}=\abs{-2}=2
\end{equation*}
\begin{esempiot}{Distanza con stessa ascissa}{distre}
	Dati i punti $A\coord{2}{4}$\,$B\coord{2}{7}$ calcolare la distanza tra $A$ e $B$
\end{esempiot}
I due punti hanno la stessa ascissa quindi: 
\begin{equation*}
	d(AB)=\abs{y_1-y_2}=\abs{4-7}=\abs{-3}=3
\end{equation*}
\begin{esempiot}{Distanza con stessa ordinata}{disdue}
	Dati i punti $A\coord{3}{-5}$\,$B\coord{-6}{-5}$ calcolare la distanza tra $A$ e $B$
\end{esempiot}
I due punti hanno la stessa ordinata quindi: 
\begin{equation*}
d(AB)=\abs{x_1-x_2}=\abs{3-(-6)}=\abs{3+6}=\abs{9}=9
\end{equation*}

\begin{esempiot}{Distanza con stessa ascissa}{disquattro}
	Dati i punti $A\coord{6}{-5}$\,$B\coord{6}{-2}$ calcolare la distanza tra $A$ e $B$
\end{esempiot}
I due punti hanno la stessa ascissa quindi: 
\begin{equation*}
d(AB)=\abs{y_1-y_2}=\abs{-5-(-2)}=\abs{-5+2}=\abs{-3}=3
\end{equation*}
\begin{esempiot}{Distanza caso generale}{dis5}
	Dati i punti $A\coord{3}{5}$\,$B\coord{4}{2}$ calcolare la distanza tra $A$ e $B$
\end{esempiot}
I due punti hanno la stessa ascissa quindi: 
\begin{align*}
	d(AB)=&\sqrt{(x_1-x_2)^2+(y_1-y_2)^2}\\
	=&\sqrt{(3-4)^2+(5-2)^2}\\
=&\sqrt{(-1)^2+(3)^2}\\
=&\sqrt{1+9}\\
=&\sqrt{10}
\end{align*}

\begin{esempiot}{Distanza caso generale}{discinque}
	Dati i punti $A\coord{2}{-4}$\,$B\coord{-5}{6}$ calcolare la distanza tra $A$ e $B$
\end{esempiot}
I due punti hanno la stessa ascissa quindi: 
\begin{align*}
d(AB)=&\sqrt{(x_1-x_2)^2+(y_1-y_2)^2}\\
=&\sqrt{(2-(-5))^2+(-4-6)^2}\\
=&\sqrt{(2+5)^2+(-10)^2}\\
=&\sqrt{49+100}\\
=&\sqrt{149}
\end{align*}


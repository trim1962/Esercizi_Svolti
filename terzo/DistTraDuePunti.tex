\chapter{Distanza tra due punti}
\label{cha:DistanzaTraduePunti}
\section{Stessa ordinata}
\begin{esempiot}{}{}
Dati i punti $A\coord{1}{2}$\,$B\coord{3}{2}$ calcolare la distanza tra $A$ e $B$
\end{esempiot}
I due punti hanno la stessa ordinata quindi:
\begin{equation}
d(AB)=\abs{x_1-x_2}=\abs{1-3}=\abs{-2}=2
\end{equation}
\begin{esempiot}{}{}
	Dati i punti $A\coord{3}{-5}$\,$B\coord{-6}{-5}$ calcolare la distanza tra $A$ e $B$
\end{esempiot}
I due punti hanno la stessa ordinata quindi: 
\begin{equation}
d(AB)=\abs{x_1-x_2}=\abs{3-(-6)}=\abs{3+6}=\abs{9}=9
\end{equation}
\section{Stessa ascissa}
\begin{esempiot}{}{}
	Dati i punti $A\coord{2}{4}$\,$B\coord{2}{7}$ calcolare la distanza tra $A$ e $B$
\end{esempiot}
I due punti hanno la stessa ascissa quindi: 
\begin{equation}
d(AB)=\abs{y_1-y_2}=\abs{4-7}=\abs{-3}=3
\end{equation}
\begin{esempiot}{}{}
	Dati i punti $A\coord{6}{-5}$\,$B\coord{6}{-2}$ calcolare la distanza tra $A$ e $B$
\end{esempiot}
I due punti hanno la stessa ascissa quindi: 
\begin{equation}
d(AB)=\abs{y_1-y_2}=\abs{-5-(-2)}=\abs{-5+2}=\abs{-3}=3
\end{equation}
\section{Caso generale}
\begin{esempiot}{}{}
	Dati i punti $A\coord{3}{5}$\,$B\coord{4}{2}$ calcolare la distanza tra $A$ e $B$
\end{esempiot}
I due punti hanno la stessa ascissa quindi: 
\begin{equation}
d(AB)=\sqrt{(x_1-x_2)^2+(y_1-y_2)^2}=\sqrt{(3-4)^2+(5-2)^2}=\sqrt{(-1)^2+(3)^2}=\sqrt{1+9}=\sqrt{10}
\end{equation}
\begin{esempiot}{}{}
	Dati i punti $A\coord{2}{-4}$\,$B\coord{-5}{6}$ calcolare la distanza tra $A$ e $B$
\end{esempiot}
I due punti hanno la stessa ascissa quindi: 
\begin{equation}
d(AB)=\sqrt{(x_1-x_2)^2+(y_1-y_2)^2}=\sqrt{(2-(-5))^2+(-4-6)^2}=\sqrt{(2+5)^2+(-10)^2}=\sqrt{49+100}=\sqrt{149}
\end{equation}

\chapter{Funzioni goniometriche usando la calcolatrice}
\label{cha:ValFunzGonioCalc}
Quando si parla di calcolatrice si parla di una calcolatrice scientifica. Devono essere presenti i tasti:

\begin{center}
	\begin{tabular}{ccc}
\tastosin&\tastocos&\tastotan \\ 
\end{tabular} 
\end{center}

In genere per le funzioni inverse si usa una combinazione di tasti \tastoshift 

\begin{center}
	\begin{tabular}{ccc}
		\tastoisin&\tastoicos&\tastoitan \\ 
	\end{tabular} 
\end{center}

La calcolatrice deve gestire i radianti, i gradi con un tasto mode \tastomode e il numero $\pi$ \tastopgreco. Utile è la presenza di un tasto \tastoans che richiama l'ultimo risultato ottenuto.
\section{Trovare valore funzione}
\subsection{Angolo in gradi}
\begin{esempiot}{}{}
Calcolare \[\sin\ang{38;28;50}\] 
\end{esempiot}
Controllare che la calcolatrice è impostata in gradi sessagesimali\index{Grado!Sessagesimale}.
Basta verificare che \testgradi In caso contrario modificare le impostazioni. 

Si inizia convertendo i gradi in forma sessadecimale\index{Grado!Sessagesimale}\index{Grado!Sessadecimale}\index{Seno}

\begin{align*}
&\phantom{=}\ang{38}+\left(\dfrac{28}{60}\right)^{\degree}+\left(\dfrac{50}{3600}\right)^{\degree }=\\
&=\ang{38}+\left(\dfrac{28\cdot60+50}{3600}\right)^{\degree}=\\
&=\ang{38}+\left(\dfrac{1680+50}{3600}\right)^{\degree}=\\
&=\ang{38}+\left(\dfrac{1730}{3600}\right)^{\degree}=\\
&=\ang[round-precision=\lungarrotandamento,round-mode=places]{38.4805556}
\end{align*}
usando la calcolatrice

\begin{center}
\begin{tabular}{ll}
\tasto{28}\tastoper\tasto{60}\tastouguale& 1680 \\ 
\tastoans\tastopiu\tasto{50}\tastouguale& 1730 \\
\tastoans\tastodiv\tasto{3600}\tastouguale& \num[round-precision=\lungarrotandamento,round-mode=places]{0.480555555} \\
\tastoans\tastopiu\tasto{38}\tastouguale&\num[round-precision=\lungarrotandamento,round-mode=places]{38.480555555} \\
\end{tabular}
\end{center} 

Infine

 \tastosin\tastoans\tastouguale e ottenere
\[\sin\ang{38;28;50}=\num[round-precision=\lungarrotandamento,round-mode=places]{0.622249007}\] 

\begin{esempiot}{}{}
	Calcolare \[\tan\ang{120;30;40}\] 
\end{esempiot}
Controllare che la calcolatrice è impostata in gradi sessagesimali\index{Grado!Sessagesimale}.
Basta verificare che \testgradi. In caso contrario modificare le impostazioni. 

Si inizia convertendo i gradi in forma sessadecimale\index{Grado!Sessagesimale}\index{Grado!Sessadecimale}\index{Tangente}

\begin{align*}
&\phantom{=}\ang{120}+\left(\dfrac{30}{60}\right)^{\degree}+\left(\dfrac{40}{3600}\right)^{\degree }=\\
&=\ang{120}+\left(\dfrac{30\cdot60+40}{3600}\right)^{\degree}=\\
&=\ang{120}+\left(\dfrac{1800+40}{3600}\right)^{\degree}=\\
&=\ang{120}+\left(\dfrac{1730}{3600}\right)^{\degree}=\\
&=\ang[round-precision=\lungarrotandamento,round-mode=places]{120.5111111}
\end{align*}

usando la calcolatrice

\begin{center}
	\begin{tabular}{ll}
		\tasto{30}\tastoper\tasto{60}\tastouguale	& 1800 \\ 
		\tastoans\tastopiu\tasto{40}\tastouguale	& 1840 \\
		\tastoans\tastodiv\tasto{3600}\tastouguale	& \num[round-precision=\lungarrotandamento,round-mode=places]{0.511111111} \\
		\tastoans\tastopiu\tasto{120}\tastouguale&\num[round-precision=\lungarrotandamento,round-mode=places]{120.511111111} \\
	\end{tabular}
\end{center} 

Infine \tastotan \tastoans\tastouguale e ottenere
\[\tan\ang{120;30;40}=\num[round-precision=\lungarrotandamento,round-mode=places]{-1.69610537}\] 
\subsection{Angolo in radianti}
\begin{esempiot}{}{}
	Calcolare \[\cos\dfrac{\pi}{4}\] 
\end{esempiot}
Controllare che la calcolatrice è impostata in radianti\index{Radianti}\index{Coseno}.
Basta verificare che 
\testradianti
 In caso contrario modificare le impostazioni.

Non resta che procedere con il calcolo
	
\begin{center}
\begin{tabular}{ll}
	\tastopgreco\tastodiv\tasto{4}\tastouguale& \num[round-precision=\lungarrotandamento,round-mode=places]{0.785398163} \\ 
\tastocos\tastoans\tastouguale	&\num[round-precision=\lungarrotandamento,round-mode=places]{0.707106781} \\ 
\end{tabular} 
\end{center}
\[\cos\dfrac{\pi}{4}=\num[round-precision=\lungarrotandamento,round-mode=places]{0.707106781}\] 
\begin{esempiot}{}{}
	Calcolare \[\tan\SI[round-precision=4,round-mode=places]{1.4589}{\radian}\] 
\end{esempiot}
Controllare che la calcolatrice è impostata in radianti\index{Radianti}\index{Tangente}.
Basta verificare che 
\testradianti
In caso contrario modificare le impostazioni.

Non resta che procedere con il calcolo

\begin{center}
	\begin{tabular}{ll}
	 \tastotan\tasto{\num[round-precision=4,round-mode=places]{1.4589}}\tastouguale& \num[round-precision=\lungarrotandamento,round-mode=places]{8.899513904}\\ 
	\end{tabular} 
\end{center}
otteniamo
 \[\tan\SI[round-precision=4,round-mode=places]{1.4589}{\radian}=\num[round-precision=\lungarrotandamento,round-mode=places]{8.899513904}\] 
 \section{Trovare l'angolo nota la funzione}
 \subsection{Angolo in gradi}
 \begin{esempiot}{}{}
 Trovare l'angolo per cui \[\cos x=\num[round-precision=\lungarrotandamento,round-mode=places]{0.778934}\]
 \end{esempiot}
Controllare che la calcolatrice è impostata in gradi sessagesimali\index{Grado!Sessagesimale}.
Basta verificare che \testgradi In caso contrario modificare le impostazioni.

Le soluzioni sono 
\[\begin{cases}
	x_1=+\alpha+k\ang{360}\\
	x_2=-\alpha+k\ang{360}\\
\end{cases}\]
Calcolo $\alpha$

\begin{center}
		\begin{tabular}{ll}
		\tastoicos\tasto{\num[round-precision=\lungarrotandamento,round-mode=places]{0.778934}}\tastouguale&\SI[round-precision=\lungarrotandamento,round-mode=places]{38.8369232}{\degree}\\
		\end{tabular}
\end{center}

Converto in gradi sessagesimali

\begin{center}		
	\begin{tabular}{ll}
		\tastoans\tastomeno\tasto{38}\tastouguale&\SI[round-precision=\lungarrotandamento,round-mode=places]{0.810314895}{\degree}\\
		\tastoans\tastoper\tasto{60}\tastouguale&\SI[round-precision=\lungarrotandamento,round-mode=places]{50.21539175}{\arcminute}\\
		\tastoans\tastomeno\tasto{50}\tastouguale&\SI[round-precision=\lungarrotandamento,round-mode=places]{0.215391754}{\arcminute}\\
		\tastoans\tastoper\tasto{60}\tastouguale&\SI[round-precision=\lungarrotandamento,round-mode=places]{12.929350526}{\arcsecond}\\
	\end{tabular} 
\end{center}
\[\alpha=\ang{38;50;12}\]
le soluzioni sono quindi
\[\begin{cases}
x_1=+\ang{38;50;12}+k\ang{360}\\
x_2=-\ang{38;50;12}+k\ang{360}\\
\end{cases}\]
 \subsection{Angolo in radianti}
 \begin{esempiot}{}{}
 	Calcolare \[\sin\alpha=\num[round-precision=\lungarrotandamento,round-mode=places]{-0.783942}\] 
 \end{esempiot}
 Controllare che la calcolatrice è impostata in radianti\index{Radianti}\index{Seno}.
 Basta verificare che 
 \testradianti
 In caso contrario modificare le impostazioni.
 
 Non resta che procedere con il calcolo
 
 \begin{center}
 	\begin{tabular}{ll}
 		\tastoisin\tasto{\num[round-precision=\lungarrotandamento,round-mode=places]{-0.783942}}\tastouguale&\num[round-precision=\lungarrotandamento,round-mode=places]{-0.900990129}\\ \tasto{2}\tastoper\tastopgreco\tastopiu\tastoans\tastouguale&\num[round-precision=\lungarrotandamento,round-mode=places]{5.382195178}\\
 	\end{tabular} 
 \end{center}
 \[\rho= \SI[round-precision=\lungarrotandamento,round-mode=places]{-0.900990129}{\radian}\]
 \[\rho= \SI[round-precision=\lungarrotandamento,round-mode=places]{5.382195178}{\radian}\]
 

\newpage
 \section{Esercizi equazioni elementari}
 \tcbstartrecording
 \begin{exercise}
Trovare l'angolo in gradi per cui $\cos x=\num[round-precision=\lungarrotandamento,round-mode=places]{-0.7548329}$
\tcblower
$\cos x=\num[round-precision=\lungarrotandamento,round-mode=places]{-0.7548329}$

 Controllare che la calcolatrice è impostata in gradi sessagesimali\index{Grado!Sessagesimale}.
 Basta verificare che \testgradi 
 
 In caso contrario modificare le impostazioni.
 
 Le soluzioni sono 
 \[\begin{cases}
 x_1=+\alpha+k\ang{360}\\
 x_2=-\alpha+k\ang{360}\\
 \end{cases}\]
 Calcolo $\alpha$
 
 \begin{center}
 	\begin{tabular}{ll}
 		\tastoicos\tasto{\num[round-precision=\lungarrotandamento,round-mode=places]{-0.7548329}}\tastouguale&\SI[round-precision=\lungarrotandamento,round-mode=places]{139.0107711}{\degree}
 	\end{tabular}
 \end{center}
 
 Converto in gradi sessagesimali
 
 \begin{center}
 	\begin{tabular}{ll}
 		\tastoans\tastomeno\tasto{139}\tastouguale&\SI[round-precision=\lungarrotandamento,round-mode=places]{0.010771083}{\degree}\\
 		\tastoans\tastoper\tasto{60}\tastouguale&\SI[round-precision=\lungarrotandamento,round-mode=places]{0.646265034}{\arcminute}\\
 		\tastoans\tastomeno\tasto{0}\tastouguale&\SI[round-precision=\lungarrotandamento,round-mode=places]{0.646265034}{\arcminute}\\
 		\tastoans\tastoper\tasto{60}\tastouguale&\SI[round-precision=\lungarrotandamento,round-mode=places]{38.77590204}{\arcsecond}\\
 	\end{tabular} 
 \end{center}
 \[\alpha=\ang{139;0;38}\]
 le soluzioni sono quindi
 \[\begin{cases}
 x_1=+\ang{139;0;38}+k\ang{360}\\
 x_2=-\ang{139;0;38}+k\ang{360}\\
 \end{cases}\]
 \end{exercise}
 \begin{exercise}
 	Trovare l'angolo in gradi per cui $\sin x=\num[round-precision=\lungarrotandamento,round-mode=places]{0.666666666}$
\tcblower
 $\sin x=\num[round-precision=\lungarrotandamento,round-mode=places]{0.666666666}$
 
 Controllare che la calcolatrice è impostata in gradi sessagesimali\index{Grado!Sessagesimale}\index{Seno}.
 
 Basta verificare che \testgradi 
 
 In caso contrario modificare le impostazioni.
 
 Le soluzioni sono 
 \[\begin{cases}
 x_1=\alpha+k\ang{360}\\
 x_2=\ang{180}-\alpha+k\ang{360}\\
 \end{cases}\]
 Calcolo $\alpha$
  \begin{center}
 	\begin{tabular}{ll}
 		\tastoisin\tasto{\num[round-precision=\lungarrotandamento,round-mode=places]{0.666666666}}\tastouguale&\SI[round-precision=\lungarrotandamento,round-mode=places]{41.8103149}{\degree}\\
 	\end{tabular}
 \end{center}
 \[\begin{cases}
 x_1=\alpha+k\ang{360}=\SI[round-precision=\lungarrotandamento,round-mode=places]{41.8103149}{\degree}+k\ang{360}\\
 x_2=\ang{180}-\alpha+k\ang{360}=\SI[round-precision=\lungarrotandamento,round-mode=places]{138.1896851}{\degree}+k\ang{360}\\
 \end{cases}\]
 
 Converto in gradi sessagesimali $x_1$
 
 \begin{center}		
 	\begin{tabular}{ll}
 		\tastoans\tastomeno\tasto{41}\tastouguale&\SI[round-precision=\lungarrotandamento,round-mode=places]{0.810314895}{\degree}\\
 		\tastoans\tastoper\tasto{60}\tastouguale&\SI[round-precision=\lungarrotandamento,round-mode=places]{48.61889374}{\arcminute}\\
 		\tastoans\tastomeno\tasto{48}\tastouguale&\SI[round-precision=\lungarrotandamento,round-mode=places]{0.68893743}{\arcminute}\\
 		\tastoans\tastoper\tasto{60}\tastouguale&\SI[round-precision=\lungarrotandamento,round-mode=places]{37.13362463}{\arcsecond}\\
 	\end{tabular} 
 \end{center}
 \[x_1=\ang{41;48;37}\]
 
 Converto in gradi sessagesimali $x_2$
 
 \begin{center}		
 	\begin{tabular}{ll}
 \tastoans\tastomeno\tasto{138}\tastouguale&\SI[round-precision=\lungarrotandamento,round-mode=places]{0.189685104}{\degree}\\
 \tastoans\tastoper\tasto{60}\tastouguale&\SI[round-precision=\lungarrotandamento,round-mode=places]{11.38110625}{\arcminute}\\
 \tastoans\tastomeno\tasto{48}\tastouguale&\SI[round-precision=\lungarrotandamento,round-mode=places]{0.381106252}{\arcminute}\\		\tastoans\tastoper\tasto{60}\tastouguale&\SI[round-precision=\lungarrotandamento,round-mode=places]{22.866375512}{\arcsecond}\\
 	\end{tabular} 
 \end{center}
 \[x_2=\ang{138;11;22}\]
 
 le soluzioni sono quindi
 \[\begin{cases}
 x_1=\ang{41;48;37}+k\ang{360}\\
 x_2=\ang{138;11;22}+k\ang{360}\\
 \end{cases}\]
 \end{exercise}
 \begin{exercise}
 	Trovare l'angolo in gradi per cui $\sin x=\num[round-precision=2,round-mode=places]{-0.75}$
\tcblower
 $\sin x=\num[round-precision=2,round-mode=places]{-0.75}$

 Controllare che la calcolatrice è impostata in gradi sessagesimali\index{Grado!Sessagesimale}\index{Seno}.
 
 Basta verificare che 
 \testgradi 
 
 In caso contrario modificare le impostazioni.
 
 Le soluzioni sono 
 \[\begin{cases}
 x_1=\alpha+k\ang{360}\\
 x_2=\ang{180}-\alpha+k\ang{360}\\
 \end{cases}\]
 Calcolo $\alpha$
 
 \begin{center}
 	\begin{tabular}{ll}
 		\tastoisin\tasto{\num[round-precision=2,round-mode=places]{-0.75}}\tastouguale&\SI[round-precision=\lungarrotandamento,round-mode=places]{-48.59037789}{\degree}
 	\end{tabular}
 \end{center}
 
 \[\begin{cases}
 x_1=\alpha+k\ang{360}=\SI[round-precision=\lungarrotandamento,round-mode=places]{-48.59037789}{\degree}+k\ang{360}=\SI[round-precision=\lungarrotandamento,round-mode=places]{311.4096221}{\degree}+k\ang{360}\\
 x_2=-\ang{180}+\alpha+k\ang{360}=\SI[round-precision=\lungarrotandamento,round-mode=places]{228.5903789}{\degree}+k\ang{360}\\
 \end{cases}\]
 
 Converto in gradi sessagesimali $x_1$
 
 \begin{center}		
 	\begin{tabular}{ll}
 		\tastoans\tastomeno\tasto{311}\tastouguale&\SI[round-precision=\lungarrotandamento,round-mode=places]{0.409622109}{\degree}\\
 		\tastoans\tastoper\tasto{60}\tastouguale&\SI[round-precision=\lungarrotandamento,round-mode=places]{24.57732655}{\arcminute}\\
 		\tastoans\tastomeno\tasto{48}\tastouguale&\SI[round-precision=\lungarrotandamento,round-mode=places]{0.577326552}{\arcminute}\\
 		\tastoans\tastoper\tasto{60}\tastouguale&\SI[round-precision=\lungarrotandamento,round-mode=places]{34.63959312}{\arcsecond}\\
 	\end{tabular} 
 \end{center}
 \[x_1=\ang{311;24;34}\]
 
 Converto in gradi sessagesimali $x_2$
 
 \begin{center}		
 	\begin{tabular}{ll}
 		\tastoans\tastomeno\tasto{228}\tastouguale&\SI[round-precision=\lungarrotandamento,round-mode=places]{0.59037789}{\degree}\\
 		\tastoans\tastoper\tasto{60}\tastouguale&\SI[round-precision=\lungarrotandamento,round-mode=places]{35.42267345}{\arcminute}\\
 		\tastoans\tastomeno\tasto{48}\tastouguale&\SI[round-precision=\lungarrotandamento,round-mode=places]{0.422673448}{\arcminute}\\
 		\tastoans\tastoper\tasto{60}\tastouguale&\SI[round-precision=\lungarrotandamento,round-mode=places]{25.36040688}{\arcsecond}\\
 	\end{tabular} 
 \end{center}
 \[x_2=\ang{228;35;25}\]
 
 le soluzioni sono quindi
 \[\begin{cases}
 x_1=\ang{311;24;34}+k\ang{360}\\
 x_2=\ang{228;35;25}+k\ang{360}\\
 \end{cases}\]
 \end{exercise}
 \begin{exercise}
Trovare l'angolo in gradi per cui $\tan x=\num[round-precision=\lungarrotandamento,round-mode=places]{1.414213562}$
\tcblower
$\tan x=\num[round-precision=\lungarrotandamento,round-mode=places]{1.414213562}$

 Controllare che la calcolatrice è impostata in gradi sessagesimali\index{Grado!Sessagesimale}.
 Basta verificare che 
 
\testgradi 
 
In caso contrario modificare le impostazioni.

Le soluzioni sono \[x_1=\alpha+k\ang{180}\]

Converto in gradi sessagesimali $x_1$
 \begin{center}
 	\begin{tabular}{ll}
 \tastoitan\tasto{\num[round-precision=\lungarrotandamento,round-mode=places]{1.414213562}}
 \tastouguale&\SI[round-precision=\lungarrotandamento,round-mode=places]{54.73561032}{\degree}\\
 \end{tabular}
\end{center}	

 Converto in gradi sessagesimali $x_1$

 \begin{center}
 	\begin{tabular}{ll}
 		 \tastoans\tastomeno\tasto{54}\tastouguale&\SI[round-precision=\lungarrotandamento,round-mode=places]{0.735610317}{\degree}\\
 		\tastoans\tastoper\tasto{60}\tastouguale&\SI[round-precision=\lungarrotandamento,round-mode=places]{44.13661903}{\arcminute}\\
 		\tastoans\tastomeno\tasto{44}\tastouguale&\SI[round-precision=\lungarrotandamento,round-mode=places]{0.136619034}{\arcminute}\\
 		\tastoans\tastoper\tasto{60}\tastouguale&\SI[round-precision=\lungarrotandamento,round-mode=places]{8.197142083}{\arcsecond}\\
 	\end{tabular} 
 \end{center}
Le soluzioni sono \[x_1=\ang{54;44;8}+k\ang{180}\]
 \end{exercise}
 \begin{exercise}
 Trovare l'angolo in gradi per cui $\tan x=\num[round-precision=\lungarrotandamento,round-mode=places]{-3.464101615}$
 \tcblower

 $\tan x=\num[round-precision=\lungarrotandamento,round-mode=places]{-3.464101615}$
 
 Controllare che la calcolatrice è impostata in gradi sessagesimali\index{Grado!Sessagesimale}.
 
 Basta verificare che 
 \testgradi 
 
 In caso contrario modificare le impostazioni.
 
 Le soluzioni sono \[x_1=\alpha+k\ang{180}\]
 
 Converto in gradi sessagesimali $x_1$
 \begin{center}
 	\begin{tabular}{ll}
 		\tastoitan\tasto{\num[round-precision=\lungarrotandamento,round-mode=places]{-3.464101615}}\tastouguale&\SI[round-precision=\lungarrotandamento,round-mode=places]{-73.89788625}{\degree}\\
 	\end{tabular}
 \end{center}	
 
 $x_1=\ang{180}\SI[round-precision=\lungarrotandamento,round-mode=places]{-73.89788625}{\degree}=\SI[round-precision=\lungarrotandamento,round-mode=places]{106.1021138}{\degree}$
 
 Converto in gradi sessagesimali $x_1$
 \begin{center}
 	\begin{tabular}{ll}
 		\tastoans\tastomeno\tasto{106}\tastouguale&\SI[round-precision=\lungarrotandamento,round-mode=places]{0.10211375}{\degree}\\
 		\tastoans\tastoper\tasto{60}\tastouguale&\SI[round-precision=\lungarrotandamento,round-mode=places]{6.126825}{\arcminute}\\
 		\tastoans\tastomeno\tasto{6}\tastouguale&\SI[round-precision=\lungarrotandamento,round-mode=places]{0.126825}{\arcminute}\\
 		\tastoans\tastoper\tasto{60}\tastouguale&\SI[round-precision=\lungarrotandamento,round-mode=places]{7.6095}{\arcsecond}\\
 	\end{tabular} 
 \end{center}
 Le soluzioni sono \[x_1=\ang{106;6;7}+k\ang{180}\]
 \end{exercise}
  \begin{exercise}[no solution]
  	Trovare l'angolo in gradi per cui $\tan x=\dfrac{\sqrt{3}}{2}$
\end{exercise}
 \begin{exercise}[no solution]
 	Trovare l'angolo in gradi per cui $\cos x=\dfrac{3}{5}$
 \end{exercise}
  \begin{exercise}[no solution]
  	Trovare l'angolo in gradi per cui $\sin x=-\dfrac{\sqrt{3}}{2}$
  	\end{exercise}
 \begin{exercise}
 Trovare l'angolo in radianti per cui $\cos x=\num[round-precision=\lungarrotandamento,round-mode=places]{-0.478973}$
  \tcblower
 $\cos x=\num[round-precision=\lungarrotandamento,round-mode=places]{-0.478973}$
 
 Controllare che la calcolatrice è impostata in radianti\index{Radianti}\index{Coseno}.
 
 Basta verificare che 
 
 \testradianti
 
 In caso contrario modificare le impostazioni.
 
 Non resta che procedere con il calcolo.
 
 Le soluzioni sono 
 \[\begin{cases}
 x_1=+\alpha+2k\pi\\
 x_2=-\alpha+2k\pi\\
 \end{cases}\]
 Calcolo $\alpha$
 \begin{center}
 	\begin{tabular}{ll}
 		\tastoicos\tasto{\num[round-precision=\lungarrotandamento,round-mode=places]{-0.4788973}}\tastouguale&\SI[round-precision=\lungarrotandamento,round-mode=places]{2.070280734}{\radian}\\ 
 	\end{tabular} 
 \end{center}
 \[\alpha= \SI[round-precision=\lungarrotandamento,round-mode=places]{2.070280734}{\radian}\]
  \[\begin{cases}
  x_1=+\SI[round-precision=\lungarrotandamento,round-mode=places]{2.070280734}+2k\pi \radian\\
  x_2=-\SI[round-precision=\lungarrotandamento,round-mode=places]{2.070280734}+2k\pi \radian\\
  \end{cases}\]
 \end{exercise}
 \begin{exercise}
 Trovare l'angolo in radianti per cui $\sin 3x=\num[round-precision=2,round-mode=places]{0.48}$
 	\tcblower
$\sin 3x=\num[round-precision=2,round-mode=places]{0.48}$ 	
 	
 Controllare che la calcolatrice è impostata in radianti\index{Radianti}\index{Seno}.
 	
 	Basta verificare che 
 	\testradianti
 	
 	In caso contrario modificare le impostazioni.
 	
 	Non resta che procedere con il calcolo.
 	
 	Le soluzioni sono 
 	\[\begin{cases}
 	x_1=\alpha+2k\pi\\
 	x_2=\pi-\alpha+2k\pi\\
 	\end{cases}\]
 	Calcolo $\alpha$
 	
 	\begin{center}
 		\begin{tabular}{ll}
 \tastoisin\tasto{\num[round-precision=2,round-mode=places]{0.48}}
 \tastouguale&\num[round-precision=\lungarrotandamento,round-mode=places]{0.500654712}\\ 
 		\end{tabular} 
 	\end{center}
 	\[\alpha= \SI[round-precision=\lungarrotandamento,round-mode=places]{0.500654712}{\radian}\]
 	\begin{align*}
  	3x_1&=\SI[round-precision=\lungarrotandamento,round-mode=places]{0.500654712}+2k\pi \radian\\
  	x_1&=\SI[round-precision=\lungarrotandamento,round-mode=places]{0.500654712}+\dfrac{2}{3}k\pi \radian\\
 	\end{align*}
 	\begin{align*}
 	3x_2&=\pi-\SI[round-precision=\lungarrotandamento,round-mode=places]{0.500654712}+2k\pi \radian\\
 	3x_2&=\SI[round-precision=\lungarrotandamento,round-mode=places]{2.640937182}+2k\pi \radian\\
 	x_2&=\SI[round-precision=\lungarrotandamento,round-mode=places]{0.880312394}+\dfrac{2}{3}k\pi \radian\\
 	\end{align*}
 Le soluzioni sono
 
\[ \begin{cases}
x_1=\SI[round-precision=\lungarrotandamento,round-mode=places]{0.500654712}+\dfrac{2}{3}k\pi \radian\\

x_2=\SI[round-precision=\lungarrotandamento,round-mode=places]{0.880312394}+\dfrac{2}{3}k\pi \radian\\
 \end{cases}\]
 \end{exercise}
  \begin{exercise}
  	Trovare l'angolo in radianti per cui $\sin 2x=\num[round-precision=3,round-mode=places]{-0.128}$
  	\tcblower
  	$\sin 2x=\num[round-precision=3,round-mode=places]{-0.128}$ 	
  	
  	Controllare che la calcolatrice è impostata in radianti\index{Radianti}\index{Seno}.
  	
  	Basta verificare che 
  	\testradianti
  	
  	In caso contrario modificare le impostazioni.
  	
  	Non resta che procedere con il calcolo.
  	
  	Le soluzioni sono 
  	\[\begin{cases}
  	x_1=\alpha+2k\pi\\
  	x_2=\pi-\alpha+2k\pi\\
  	\end{cases}\]
  	Calcolo $\alpha$
  	
  	\begin{center}
  		\begin{tabular}{ll}
  			\tastoisin\tasto{\num[round-precision=3,round-mode=places]{-0.128}}
  			\tastouguale&\num[round-precision=\lungarrotandamento,round-mode=places]{-0.128352127}	
  		\end{tabular} 
  	\end{center}
  	\[\alpha=2\pi+ \SI[round-precision=\lungarrotandamento,round-mode=places]{-0.128352127}{\radian}\]
  	\begin{align*}
  	2x_1&=\SI[round-precision=\lungarrotandamento,round-mode=places]{6.154833179}+2k\pi \radian\\
  	x_1&=\SI[round-precision=\lungarrotandamento,round-mode=places]{3.205768717}+k\pi \radian\\
  	\end{align*}
  	\begin{align*}
  	2x_2&=\pi+\SI[round-precision=\lungarrotandamento,round-mode=places]{3.269944781}+2k\pi \radian\\
  	2x_2&=\SI[round-precision=\lungarrotandamento,round-mode=places]{2.640937182}+2k\pi \radian\\
  	x_2&=\SI[round-precision=\lungarrotandamento,round-mode=places]{1.63497239}+k\pi \radian\\
  	\end{align*}
  	Le soluzioni sono
  	
  	\[ \begin{cases}
  	x_1&=\SI[round-precision=\lungarrotandamento,round-mode=places]{3.205768717}+k\pi \radian\\
  	
  		x_2&=\SI[round-precision=\lungarrotandamento,round-mode=places]{1.63497239}+k\pi \radian\\ \radian\\
  	\end{cases}\]
  \end{exercise}
 \begin{exercise}
 	Trovare l'angolo in radianti per cui $\cos 2x=\num[round-precision=3,round-mode=places]{0.128}$
 	\tcblower
 	$\cos 2x=\num[round-precision=3,round-mode=places]{0.128}$ 	
 	
 	Controllare che la calcolatrice è impostata in radianti\index{Radianti}\index{Coseno}.
 	
 	Basta verificare che 
 	\testradianti
 	
 	In caso contrario modificare le impostazioni.
 	
 	Non resta che procedere con il calcolo.
 	
 	Le soluzioni sono 
 	\[\begin{cases}
 	x_1=\alpha+2k\pi\\
 	x_2=-\alpha+2k\pi\\
 	\end{cases}\]
 	Calcolo $\alpha$
 	
 	\begin{center}
 		\begin{tabular}{ll}
 			\tastoicos\tasto{\num[round-precision=3,round-mode=places]{0.128}}
 			\tastouguale&\num[round-precision=\lungarrotandamento,round-mode=places]{1.442444199}	
 		\end{tabular} 
 	\end{center}
 	\[\alpha=2\pi+ \SI[round-precision=\lungarrotandamento,round-mode=places]{1.442444199}{\radian}\]
 	\begin{align*}
 	2x_1&=\SI[round-precision=\lungarrotandamento,round-mode=places]{1.442444199}+2k\pi \radian\\
 	x_1&=\SI[round-precision=\lungarrotandamento,round-mode=places]{0.721222099}+k\pi \radian\\
 	\end{align*}
% 	\begin{align*}
% 	2x_2&=\pi+\SI[round-precision=\lungarrotandamento,round-mode=places]{3.269944781}+2k\pi \radian\\
% 	2x_2&=\SI[round-precision=\lungarrotandamento,round-mode=places]{2.640937182}+2k\pi \radian\\
% 	x_2&=\SI[round-precision=\lungarrotandamento,round-mode=places]{1.63497239}+k\pi \radian\\
% 	\end{align*}
 	Le soluzioni sono
 	
 	\[ \begin{cases}
 	x_1&=+\SI[round-precision=\lungarrotandamento,round-mode=places]{1.442444199}+k\pi \radian\\
 	
 	x_2&=-\SI[round-precision=\lungarrotandamento,round-mode=places]{1.442444199}+k\pi \radian\\ \radian\\
 	\end{cases}\]
 \end{exercise}
 	\tcbstoprecording
 	\newpage
 	\section{Soluzioni equazioni goniometriche elementari}
 	\tcbinputrecords
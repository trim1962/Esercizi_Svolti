\chapter{Funzioni goniometriche usando la calcolatrice}
\label{cha:ValFunzGonioCalc}
Quando si parla di calcolatrice si parla di una calcolatrice scientifica. Devono essere presenti i tasti:

\begin{center}
	\begin{tabular}{ccc}
\tastosin&\tastocos&\tastotan \\ 
\end{tabular} 
\end{center}

In genere per le funzioni inverse si usa una combinazione di tasti \tastoshift  

\begin{center}
	\begin{tabular}{ccc}
		\tastoisin&\tastoicos&\tastoitan \\ 
	\end{tabular} 
\end{center}

La calcolatrice deve gestire i radianti, i gradi con un tasto mode \tastomode e il numero $\pi$ \tastopgreco. Utile è la presenza di un tasto \tastoans che richiama l'ultimo risultato ottenuto.
\section{Angolo in gradi}
\begin{esempiot}{}{}
Calcolare  \[\sin\ang{38;28;50}\] 
\end{esempiot}
Controllare che la calcolatrice è impostata in gradi sessagesimali\index{Grado!Sessagesimale}.
Basta verificare che \tastosin \tasto{90}\tastouguale $1$. In caso contrario modificare le impostazioni. 

Si inizia convertendo  i gradi in forma sessadecimale\index{Grado!Sessagesimale}\index{Grado!Sessadecimale}\index{Seno}

\begin{align*}
&\phantom{=}\ang{38}+\left(\dfrac{28}{60}\right)^{\degree}+\left(\dfrac{50}{3600}\right)^{\degree }=\\
&=\ang{38}+\left(\dfrac{28\cdot60+50}{3600}\right)^{\degree}=\\
&=\ang{38}+\left(\dfrac{1680+50}{3600}\right)^{\degree}=\\
&=\ang{38}+\left(\dfrac{1730}{3600}\right)^{\degree}=\\
&=\ang[round-precision=6,round-mode=places]{38.4805556}
\end{align*}
usando la calcolatrice

\begin{center}
\begin{tabular}{ll}
\tasto{28}\tastopper\tasto{60}\tastouguale	& 1680 \\ 
\tastoans\tastopiu\tasto{50}\tastouguale	& 1730 \\
\tastoans\tastodiv\tasto{3600}\tastouguale	& \num[round-precision=6,round-mode=places]{0.480555555} \\
\tastoans\tastopiu\tasto{38}\tastouguale&\num[round-precision=6,round-mode=places]{38.480555555} \\
\end{tabular}
\end{center} 

Infine

 \tastosin \tastoans\tastouguale e ottenere
\[\sin\ang{38;28;50}=\num[round-precision=6,round-mode=places]{0.622249007}\] 

\begin{esempiot}{}{}
	Calcolare  \[\tan\ang{120;30;40}\] 
\end{esempiot}
Controllare che la calcolatrice è impostata in gradi sessagesimali\index{Grado!Sessagesimale}.
Basta verificare che  \tastosin \tasto{90}\tastouguale $1$. In caso contrario modificare le impostazioni. 

Si inizia convertendo  i gradi in forma sessadecimale\index{Grado!Sessagesimale}\index{Grado!Sessadecimale}\index{Tangente}

\begin{align*}
&\phantom{=}\ang{120}+\left(\dfrac{30}{60}\right)^{\degree}+\left(\dfrac{40}{3600}\right)^{\degree }=\\
&=\ang{120}+\left(\dfrac{30\cdot60+40}{3600}\right)^{\degree}=\\
&=\ang{120}+\left(\dfrac{1800+40}{3600}\right)^{\degree}=\\
&=\ang{120}+\left(\dfrac{1730}{3600}\right)^{\degree}=\\
&=\ang[round-precision=6,round-mode=places]{120.5111111}
\end{align*}

usando la calcolatrice

\begin{center}
	\begin{tabular}{ll}
		\tasto{30}\tasto{$\times$}\tasto{60}\tastouguale	& 1800 \\ 
		\tastoans\tastopiu\tasto{40}\tastouguale	& 1840 \\
		\tastoans\tastodiv\tasto{3600}\tastouguale	& \num[round-precision=6,round-mode=places]{0.511111111} \\
		\tastoans\tastopiu\tasto{120}\tastouguale&\num[round-precision=6,round-mode=places]{120.511111111} \\
	\end{tabular}
\end{center} 

Infine \tastotan \tastoans\tastouguale e ottenere
\[\tan\ang{120;30;40}=\num[round-precision=6,round-mode=places]{-1.69610537}\] 
\section{Angolo in radianti}
\begin{esempiot}{}{}
	Calcolare  \[\cos\dfrac{\pi}{4}\] 
\end{esempiot}
Controllare che la calcolatrice è impostata in radianti\index{Radianti}\index{Coseno}.
Basta verificare che 
\begin{center}
\begin{tabular}{ll}
\tastopgreco\tastodiv\tasto{2}\tastouguale	&\num[round-precision=6,round-mode=places]{1.570796327}  \\ 
	\tastosin\tastoans& 1 \\ 
\end{tabular} 
\end{center}
 In caso contrario modificare le impostazioni.
\begin{center}
	
	Non resta che procedere con il calcolo
	
\begin{tabular}{ll}
	\tastopgreco\tastodiv\tasto{4}\tastouguale& \num[round-precision=6,round-mode=places]{0.785398163}  \\ 
\tastocos\tastoans	&\num[round-precision=6,round-mode=places]{0.707106781}   \\ 
\end{tabular} 
\end{center}
\[\cos\dfrac{\pi}{4}=\num[round-precision=6,round-mode=places]{0.707106781}\] 
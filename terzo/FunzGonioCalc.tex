\chapter{Funzioni goniometriche usando la calcolatrice}
\label{cha:ValFunzGonioCalc}
Quando si parla di calcolatrice si parla di una calcolatrice scientifica. Devono essere presenti i tasti:

\begin{center}
	\begin{tabular}{ccc}
\tastosin&\tastocos&\tastotan \\ 
\end{tabular} 
\end{center}

In genere per le funzioni inverse si usa una combinazione di tasti \tastoshift  

\begin{center}
	\begin{tabular}{ccc}
		\tastoisin&\tastoicos&\tastoitan \\ 
	\end{tabular} 
\end{center}

La calcolatrice deve gestire i radianti, i gradi con un tasto mode \tastomode e il numero $\pi$ \tastopgreco. Utile è la presenza di un tasto \tastoans che richiama l'ultimo risultato ottenuto.
\section{Trovare valore funzione}
\subsection{Angolo in gradi}
\begin{esempiot}{}{}
Calcolare  \[\sin\ang{38;28;50}\] 
\end{esempiot}
Controllare che la calcolatrice è impostata in gradi sessagesimali\index{Grado!Sessagesimale}.
Basta verificare che \testgradi In caso contrario modificare le impostazioni. 

Si inizia convertendo  i gradi in forma sessadecimale\index{Grado!Sessagesimale}\index{Grado!Sessadecimale}\index{Seno}

\begin{align*}
&\phantom{=}\ang{38}+\left(\dfrac{28}{60}\right)^{\degree}+\left(\dfrac{50}{3600}\right)^{\degree }=\\
&=\ang{38}+\left(\dfrac{28\cdot60+50}{3600}\right)^{\degree}=\\
&=\ang{38}+\left(\dfrac{1680+50}{3600}\right)^{\degree}=\\
&=\ang{38}+\left(\dfrac{1730}{3600}\right)^{\degree}=\\
&=\ang[round-precision=6,round-mode=places]{38.4805556}
\end{align*}
usando la calcolatrice

\begin{center}
\begin{tabular}{ll}
\tasto{28}\tastoper\tasto{60}\tastouguale	& 1680 \\ 
\tastoans\tastopiu\tasto{50}\tastouguale	& 1730 \\
\tastoans\tastodiv\tasto{3600}\tastouguale	& \num[round-precision=6,round-mode=places]{0.480555555} \\
\tastoans\tastopiu\tasto{38}\tastouguale&\num[round-precision=6,round-mode=places]{38.480555555} \\
\end{tabular}
\end{center} 

Infine

 \tastosin \tastoans\tastouguale e ottenere
\[\sin\ang{38;28;50}=\num[round-precision=6,round-mode=places]{0.622249007}\] 

\begin{esempiot}{}{}
	Calcolare  \[\tan\ang{120;30;40}\] 
\end{esempiot}
Controllare che la calcolatrice è impostata in gradi sessagesimali\index{Grado!Sessagesimale}.
Basta verificare che  \testgradi. In caso contrario modificare le impostazioni. 

Si inizia convertendo  i gradi in forma sessadecimale\index{Grado!Sessagesimale}\index{Grado!Sessadecimale}\index{Tangente}

\begin{align*}
&\phantom{=}\ang{120}+\left(\dfrac{30}{60}\right)^{\degree}+\left(\dfrac{40}{3600}\right)^{\degree }=\\
&=\ang{120}+\left(\dfrac{30\cdot60+40}{3600}\right)^{\degree}=\\
&=\ang{120}+\left(\dfrac{1800+40}{3600}\right)^{\degree}=\\
&=\ang{120}+\left(\dfrac{1730}{3600}\right)^{\degree}=\\
&=\ang[round-precision=6,round-mode=places]{120.5111111}
\end{align*}

usando la calcolatrice

\begin{center}
	\begin{tabular}{ll}
		\tasto{30}\tastoper\tasto{60}\tastouguale	& 1800 \\ 
		\tastoans\tastopiu\tasto{40}\tastouguale	& 1840 \\
		\tastoans\tastodiv\tasto{3600}\tastouguale	& \num[round-precision=6,round-mode=places]{0.511111111} \\
		\tastoans\tastopiu\tasto{120}\tastouguale&\num[round-precision=6,round-mode=places]{120.511111111} \\
	\end{tabular}
\end{center} 

Infine \tastotan \tastoans\tastouguale e ottenere
\[\tan\ang{120;30;40}=\num[round-precision=6,round-mode=places]{-1.69610537}\] 
\subsection{Angolo in radianti}
\begin{esempiot}{}{}
	Calcolare  \[\cos\dfrac{\pi}{4}\] 
\end{esempiot}
Controllare che la calcolatrice è impostata in radianti\index{Radianti}\index{Coseno}.
Basta verificare che 
\testradianti
 In caso contrario modificare le impostazioni.

Non resta che procedere con il calcolo
	
\begin{center}
\begin{tabular}{ll}
	\tastopgreco\tastodiv\tasto{4}\tastouguale& \num[round-precision=6,round-mode=places]{0.785398163}  \\ 
\tastocos\tastoans\tastouguale	&\num[round-precision=6,round-mode=places]{0.707106781}   \\ 
\end{tabular} 
\end{center}
\[\cos\dfrac{\pi}{4}=\num[round-precision=6,round-mode=places]{0.707106781}\] 
\begin{esempiot}{}{}
	Calcolare  \[\tan\SI[round-precision=4,round-mode=places]{1.4589}{\radian}\] 
\end{esempiot}
Controllare che la calcolatrice è impostata in radianti\index{Radianti}\index{Tangente}.
Basta verificare che 
\testradianti
In caso contrario modificare le impostazioni.

Non resta che procedere con il calcolo

\begin{center}
	\begin{tabular}{ll}
	  \tastotan\tasto{\num[round-precision=4,round-mode=places]{1.4589}}\tastouguale& \num[round-precision=6,round-mode=places]{8.899513904}\\ 
	\end{tabular} 
\end{center}
otteniamo
 \[\tan\SI[round-precision=4,round-mode=places]{1.4589}{\radian}=\num[round-precision=6,round-mode=places]{8.899513904}\] 
 \section{Trovare l'angolo nota la funzione}
 \subsection{Angolo in gradi}
 \begin{esempiot}{}{}
 Trovare l'angolo per cui \[\cos\alpha=\num[round-precision=6,round-mode=places]{0.778934}\]
 \end{esempiot}
Controllare che la calcolatrice è impostata in gradi  sessagesimali\index{Grado!Sessagesimale}.
Basta verificare che  \testgradi In caso contrario modificare le impostazioni.

\begin{center}
	\begin{tabular}{ll}
		\tastoicos\tasto{\num[round-precision=6,round-mode=places]{0.778934}}\tastouguale&\num[round-precision=6,round-mode=places]{38.8369232}\\  \tastoans\tastomeno\tasto{38}\tastouguale&\num[round-precision=6,round-mode=places]{0.836923195}\\
		\tastoans\tastoper\tasto{60}\tastouguale&\num[round-precision=6,round-mode=places]{50.21539175}\\
		\tastoans\tastomeno\tasto{50}\tastouguale&\num[round-precision=6,round-mode=places]{0.215391754}\\
		\tastoans\tastoper\tasto{60}\tastouguale&\num[round-precision=6,round-mode=places]{12.929350526}\\
	\end{tabular} 
\end{center}
\[\alpha=\ang{38;50;12}\]
 \begin{esempiot}{}{}
 	Trovare l'angolo per cui \[\tan\alpha=\num[round-precision=6,round-mode=places]{1.414213562}\]
 \end{esempiot}
 Controllare che la calcolatrice è impostata in gradi  sessagesimali\index{Grado!Sessagesimale}.
 Basta verificare che  \testgradi In caso contrario modificare le impostazioni.
 
 \begin{center}
 	\begin{tabular}{ll}
 		\tastoitan\tasto{\num[round-precision=6,round-mode=places]{1.414213562}}\tastouguale&\num[round-precision=6,round-mode=places]{54.73561032}\\  \tastoans\tastomeno\tasto{54}\tastouguale&\num[round-precision=6,round-mode=places]{0.735610317}\\
 		\tastoans\tastoper\tasto{60}\tastouguale&\num[round-precision=6,round-mode=places]{44.13661903}\\
 		\tastoans\tastomeno\tasto{44}\tastouguale&\num[round-precision=6,round-mode=places]{0.136619034}\\
 		\tastoans\tastoper\tasto{60}\tastouguale&\num[round-precision=6,round-mode=places]{8.197142083}\\
 	\end{tabular} 
 \end{center}
 \[\alpha=\ang{54;44;8}\]
 \subsection{Angolo in radianti}
 \begin{esempiot}{}{}
 	Calcolare  \[\sin\alpha=\SI[round-precision=6,round-mode=places]{-0.783942}{\radian}\] 
 \end{esempiot}
 Controllare che la calcolatrice è impostata in radianti\index{Radianti}\index{Seno}.
 Basta verificare che 
 \testradianti
 In caso contrario modificare le impostazioni.
 
 Non resta che procedere con il calcolo
 
 \begin{center}
 	\begin{tabular}{ll}
 		\tastoisin\tasto{\num[round-precision=6,round-mode=places]{-0.783942}}\tastouguale&\num[round-precision=6,round-mode=places]{-0.900990129}\\  \tasto{2}\tastoper\tastopgreco\tastopiu\tastoans\tastouguale&\num[round-precision=6,round-mode=places]{5.382195178}\\
 	\end{tabular} 
 \end{center}
 \[\rho= \SI[round-precision=6,round-mode=places]{-0.900990129}{\radian}\]
 \[\rho= \SI[round-precision=6,round-mode=places]{5.382195178}{\radian}\]
  \begin{esempiot}{}{}
  	Calcolare  \[\cos\alpha=\SI[round-precision=6,round-mode=places]{-0.478973}{\radian}\] 
  \end{esempiot}
  Controllare che la calcolatrice è impostata in radianti\index{Radianti}\index{Seno}.
  Basta verificare che 
  \testradianti
  In caso contrario modificare le impostazioni.
  
  Non resta che procedere con il calcolo
  
  \begin{center}
  	\begin{tabular}{ll}
  		\tastoicos\tasto{\num[round-precision=6,round-mode=places]{-0.4788973}}\tastouguale&\num[round-precision=6,round-mode=places]{2.070280734}\\  
  	\end{tabular} 
  \end{center}
  \[\rho= \SI[round-precision=6,round-mode=places]{2.070280734}{\radian}\]
\chapter{Esercizi di riepilogo geometria analitica}
\section{Distanza}
\tcbstartrecording
\begin{exercise}
Calcola il perimetro del triangolo che ha per vertici $A\coord{-3}{2}$, $B\coord{3}{2}$ e $C\coord{0}{-3}$
\tcblower
Calcolo la distanza tra $AB$ come con l'\cref{exa:disuno} 
quindi \[d(AB)=\abs{-3-3}=\abs{-6}=6\] Ora calcoliamo la distanza tra $BC$ come con \cref{exa:discinque} 
\begin{align*}
d(BC)=&\sqrt{(3)^2+(2+3)^2}\\
=&\sqrt{34}
\end{align*}
Ora calcoliamo la distanza tra $AC$ come con \cref{exa:discinque} 
\begin{align*}
d(AC)=&\sqrt{(-3)^2+(2+3)^2}\\
=&\sqrt{34}
\end{align*}
Conseguentemente il perimetro è
\[2P=6+\sqrt{34}+\sqrt{34}=6+2\sqrt{34}\]
\begin{center}
	\includestandalone[width=.5\textwidth]{terzo/grafici/retta_dis_11}
	\captionof{figure}{Perimetro triangolo}\label{fig:EsRieDistanza11}
\end{center}
\end{exercise}
\begin{exercise}
	Calcola il perimetro della che ha per vertici $A\coord{-3}{1}$, $B\coord{-1}{4}$, $C\coord{-6}{4}$ e  $D\coord{-8}{1}$
	\tcblower
	Calcoliamo la lunghezza tra $AB$ come con \cref{exa:discinque} 
	\begin{align*}
		d(AB)=&\sqrt{(-3+1)^2+(1-4)^2}\\
		=&\sqrt{4+9}\\
		=&\sqrt{13}\\
	\end{align*}
	Calcolo la distanza tra $CB$ come con l'\cref{exa:disuno} 
	quindi \[d(CB)=\abs{-1+6}=\abs{5}=5\]
	Ora calcoliamo la distanza tra $CD$ come con \cref{exa:discinque} 
	\begin{align*}
		d(CD)=&\sqrt{(-6+8)^2+(4-1)^2}\\
			=&\sqrt{4+9}\\
		=&\sqrt{13}
	\end{align*}
	Calcolo la distanza tra $DA$ come con l'\cref{exa:disuno} 
	quindi \[d(DA)=\abs{-3+8}=\abs{5}=5\]
	Conseguentemente il perimetro è
	\[2P=5+\sqrt{13}+5+\sqrt{13}=10+2\sqrt{13}\]
	\begin{center}
		\includestandalone[width=.5\textwidth]{terzo/grafici/retta_dis_12}
		\captionof{figure}{Perimetro}\label{fig:EsRieDistanza12}
	\end{center}
\end{exercise}
\tcbstoprecording
\newpage
\section{Soluzioni esercizi di riepilogo}
\tcbinputrecords							
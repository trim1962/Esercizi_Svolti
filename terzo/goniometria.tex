\chapter{Goniometria}
\label{cha:goniometriaEss}
\section{Trovare seno coseno e tangente}\index{Seno}\index{Coseno}\index{Tangente}
\begin{esempiot}{}{exemplum1}
	Se $sin\alpha=\dfrac{3}{5}$ con $\dfrac{\pi}{2}\leq\alpha\leq\pi$ determinare coseno e tangente.
\end{esempiot}
\begin{figure}
	\centering
	\includestandalone[width=.5\textwidth]{terzo/grafici/EquaElementareSeno1}
	\captionsetup{format=esempio}
	\caption{Seno noto}\label{fig:esempio1}
\end{figure}
Se $\sin\alpha=\dfrac{3}{5}$ disegno la figura\nobs\vref{fig:esempio1}. Con il vincolo $\dfrac{\pi}{2}\leq\alpha\leq\pi$ il Punto $P$ è nel secondo quadrante, di conseguenza il coseno è negativo, come è negativa la tangente.
\begin{align*}
\sin\alpha&=\dfrac{3}{5}\\
\cos\alpha&=-\sqrt{1-\sin^2\alpha}=\\
&=-\sqrt{1-\dfrac{9}{25}}=\\
&=-\sqrt{\dfrac{25-9}{25}}=\\
&=-\sqrt{\dfrac{16}{25}}=\\
&=-\dfrac{4}{5}\\
\end{align*}
Di conseguenza la tangente è:
\[\tan\alpha=\dfrac{\sin\alpha}{\cos\alpha}=\dfrac{\dfrac{3}{5}}{-\dfrac{4}{5}}=\dfrac{3}{5}\cdot\left(-\dfrac{5}{4}\right)=-\dfrac{3}{4}\]
\begin{figure}
	\centering
	\includestandalone[width=.5\textwidth]{terzo/grafici/EquaElementareCoseno1}
	\captionsetup{format=esempio}
	\caption{Coseno noto}\label{fig:esempio2}
\end{figure}
\begin{esempiot}{}{}
	Se $cos\alpha=\dfrac{7}{9}$ con $\dfrac{3\pi}{2}\leq\alpha\leq 2\pi$ determinare seno e tangente.
\end{esempiot}
L'angolo è nel quarto quadrante, quindi seno e tangente sono negativi
\begin{align*}
\cos\alpha&=\dfrac{7}{9}\\
\sin\alpha&=-\sqrt{1-\cos^2\alpha}=\\
&=-\sqrt{1-\dfrac{49}{81}}=\\
&=-\sqrt{\dfrac{81-49}{81}}=\\
&=-\sqrt{\dfrac{32}{81}}=\\
&=-\dfrac{4\sqrt{2}}{9}\\
\end{align*}
\[\tan\alpha=\dfrac{\sin\alpha}{\cos\alpha}=\dfrac{-\dfrac{4\sqrt{2}}{9}}{\dfrac{7}{9}}=-\dfrac{4\sqrt{2}}{9}\cdot\dfrac{9}{7}=-\dfrac{4\sqrt{2}}{7}\]
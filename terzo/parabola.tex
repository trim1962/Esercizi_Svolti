\chapter{Parabola}
\label{cha:parabola}
\section{Disegnare una parabola}
\begin{esempiot}{Parabola nota}{parabola1}
	Disegnare il grafico della retta $y=4x^2-18x+18$
	\end{esempiot}
	La parabola ha $a>0$ quindi ha la concavità rivolta verso l'alto. La parabola ha l'asse parallelo all'asse $y$ ha quindi equazione\[ x=-\dfrac{b}{2a}=-\dfrac{-18}{2\cdot 4}=\dfrac{18}{8}=\dfrac{9}{4}\]
	Per trovare l'ordinata del vertice $V$ sostituisco l'ascissa dell'asse nell'equazione della parabola
	\begin{align*}
y=&4x^2-18x+18\\
=&4\left( \dfrac{9}{4}\right)^2-18\dfrac{9}{4}+18\\
=&4\dfrac{81}{16}-18\dfrac{9}{4}+18\\
=&\dfrac{81}{4}-\dfrac{162}{4}+18\\
=&\dfrac{81-162+72}{4}\\
=&-\dfrac{9}{4}
	\end{align*}
Passaggio per punti
\begin{align*}
x&=1\\
y=&4x^2-18x+18\\
=&4(1)^2-18\cdot1 +18\\
=&4
\end{align*}

\begin{align*}
x&=2\\
y=&4x^2-18x+18\\
=&4(2)^2-18\cdot2 +18\\
=&-2
\end{align*}

Ricapitolando
		\begin{tabular}{c|c}
			x & y\\
			\hline 
			1& 4 \\ 
			2&-2  \\ 
		\end{tabular}
			
$A\coord{1}{4}$ $B\coord{2}{-2}$ Per simmetria rispetto all'asse ottengo altri due punti.
\begin{center}
	\includestandalone[width=.5\textwidth]{terzo/grafici/parabola1}
	\captionof{figure}{Grafico parabola}\label{fig:disegnoparabola1}
\end{center}
\begin{esempiot}{Parabola nota}{parabola2}
	Disegnare il grafico della retta $y=-2x^2+8x-6$
\end{esempiot}
\begin{center}
	\includestandalone[width=.5\textwidth]{terzo/grafici/parabola2}
	\captionof{figure}{Grafico parabola}\label{fig:disegnoparabola2}
\end{center}
	

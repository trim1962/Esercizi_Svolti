\chapter{Retta}
\label{cha:retta}
\section{Forma esplicita}
\begin{esempiot}{retta per due punti}{esempio1}
	Disegnare il grafico della retta $y=x+1$
\end{esempiot}
Costruisco la tabella a doppia entrata 
\begin{tabular}{c|c}
	 x & y\\
	\hline 
  &  \\ 
	 &  \\ 
\end{tabular}
 
Si danno dei valori alla $x$ e si completa la tabella
$y=1+1=2$

\begin{tabular}{c|c}
	x & y\\
	\hline 
1	& 2 \\ 
	&  \\ 
\end{tabular}

Si da un altro valore alla $x$ si ottiene 

$y=2+1=3$

\begin{tabular}{c|c}
	x & y\\
	\hline 
	1	& 2 \\ 
	2& 3 \\ 
\end{tabular}

%\begin{figure}
%	\centering
\includestandalone[width=.6\textwidth]{terzo/grafici/retta_dis_1}
%	\caption{Grafico retta per due punti}\label{fig:disegnoretta1}
%\end{figure}
\cref{exa:esempio1} 

\Cref{exa:esempio1}
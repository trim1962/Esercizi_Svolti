\chapter{Retta}
\label{cha:retta}
\section{Disegnare una retta}
\begin{esempiot}{Retta nota}{retta1}
	Disegnare il grafico della retta $y=x+1$
\end{esempiot}
\begin{enumerate}
	\item Costruisco la tabella a doppia entrata 
		\begin{tabular}{c|c}
		x & y\\
		\hline 
		&  \\ 
		&  \\ 
	\end{tabular}
	\item Si attribuisce  un valore alla $x$ e si completa la tabella
	$y=1+1=2$
	\begin{tabular}{c|c}
		x & y\\
		\hline 
		1	& 2 \\ 
		&  \\ 
	\end{tabular}
	\item Si da un altro valore alla $x$ si ottiene 
	$y=2+1=3$
	\begin{tabular}{c|c}
		x & y\\
		\hline 
		1	& 2 \\ 
		2& 3 \\ 
	\end{tabular}
	\item Ottengo le coppie $A\coord{1}{2}$\,$B\coord{2}{3}$ che unite formano il grafico~\vref{fig:disegnoretta1}
\end{enumerate}
\begin{center}
\includestandalone[width=.6\textwidth]{terzo/grafici/retta_dis_1}
\captionof{figure}{Due punti una retta}\label{fig:disegnoretta1}
\end{center}
\begin{esempiot}{Retta nota}{retta2}
	Disegnare il grafico della retta $y=-2x+1$
\end{esempiot}
\begin{enumerate}
	\item Costruisco la tabella a doppia entrata 
	\begin{tabular}{c|c}
		x & y\\
		\hline 
		&  \\ 
		&  \\ 
	\end{tabular}
	\item Si attribuisce  un valore alla $x$ e si completa la tabella
	$y=-2+1=-1$
	\begin{tabular}{c|c}
		x & y\\
		\hline 
		1	& -1 \\ 
		&  \\ 
	\end{tabular}
	\item Si da un altro valore alla $x$ si ottiene 
	$y=-2(-1)+1=2+1=3$
	\begin{tabular}{c|c}
		x & y\\
		\hline 
		1	& 2 \\ 
		-1& 3 \\ 
	\end{tabular}
	\item Ottengo le coppie $A\coord{1}{2}$\,$B\coord{2}{3}$ che unite formano il grafico~\vref{fig:disegnoretta2}
\end{enumerate}
\begin{center}
	\includestandalone[width=.5\textwidth]{terzo/grafici/retta_dis_2}
	\captionof{figure}{Due punti una retta}\label{fig:disegnoretta2}
\end{center}
\begin{esempiot}{Retta nota parallela all'asse $x$}{retta3}
	Disegnare il grafico della retta $y=2$
\end{esempiot}
\begin{enumerate}
	\item Costruisco la tabella a doppia entrata 
	\begin{tabular}{c|c}
		x & y\\
		\hline 
		1&2\\ 
		2&2\\ 
	\end{tabular}
	\item Ottengo le coppie $A\coord{1}{2}$\,$B\coord{2}{2}$ che unite formano il grafico~\vref{fig:disegnoretta3}
\end{enumerate}
\begin{center}
	\includestandalone[width=.6\textwidth]{terzo/grafici/retta_dis_3}
	\captionof{figure}{Due punti una retta}\label{fig:disegnoretta3}
\end{center}
\begin{esempiot}{Retta nota parallela asse $y$}{retta4}
	Disegnare il grafico della retta $x=2$
\end{esempiot}
\begin{enumerate}
	\item Costruisco la tabella a doppia entrata 
	\begin{tabular}{c|c}
		x & y\\
		\hline 
		2&1\\ 
		2&2\\ 
	\end{tabular}
	\item Ottengo le coppie $A\coord{2}{1}$\,$B\coord{2}{2}$ che unite formano il grafico~\vref{fig:disegnoretta4}
\end{enumerate}
\begin{center}
	\includestandalone[width=.6\textwidth]{terzo/grafici/retta_dis_4}
	\captionof{figure}{Due punti una retta}\label{fig:disegnoretta4}
\end{center}
\section{Passaggio per un punto}
\begin{esempiot}{Verificare se una retta passa per un punto}{punto1}
Data la retta $y=3x+5$ Verificare se la retta passa per i punti $A\coord{1}{8}$\,$B\coord{-1}{3}$
\end{esempiot}
Verifico se la retta passa per il punto $A\coord{1}{8}$ 
\begin{enumerate}
	\item Sostituisco $A\coord{1}{8}$ nella retta $y=3x+5$ 
	\item Ottengo  \begin{tabular}{rll}
		$8=$&$+3\cdot 1$ &$+5$  \\ 
		$8=$&$3+5$  \\ 
		$8=$&$8$  \\ 
	\end{tabular}
\end{enumerate}	
	La retta passa per $A$
	
Verifico se la retta passa per il punto $B\coord{-1}{3}$
\begin{enumerate}
	\item Sostituisco $B\coord{-1}{3}$ nella retta $y=3x+5$ 
	\item Ottengo  \begin{tabular}{rll}
		$3=$&$+3\cdot (-1)$ &$+5$  \\ 
		$3=$&$-3+5$  \\ 
		$8=$&$2$  \\ 
	\end{tabular}
\end{enumerate}	
La retta non passa per $B$ 
\section{Retta per due punti}
\begin{esempiot}{Dati due punti trovare l'equazione della retta}{retta5}
Data la coppia di punti	$A\coord{1}{3}$\,$B\coord{-1}{1}$ trovare l'equazione della retta che passa per questi punti.
\end{esempiot}
\begin{enumerate}
	\item Considero la retta generica $y=mx+q$
	\item Passaggio per $A\coord{1}{3}$ ottengo $3=1\cdot m+q$
	\item Passaggio per $B\coord{-1}{1}$ ottengo $1=-1\cdot m+q$
	\item Allineo i due risultati e sottraggo
	\begin{tabular}{rll}
	$3=$&$+1\cdot m$ &$+q$  \\ 
	$1=$&$-1\cdot m$ &$+q$  \\ 
	\hline  $2=$&$+2\cdot m$& $0$ \\ 
	\end{tabular} 
	\item Semplifico e ottengo $m=1$
	\item Prendo una delle precedenti relazioni e sostituisco il valore di $m=1$ trovato
	\item \begin{tabular}{rll}
			$3=$&$+1\cdot 1$ &$+q$  \\ 
			$3=$&$+1$ &$+q$  \\ 
			$3-1=$& $+q$  \\ 
			$q=$&$+2$ 
		\end{tabular} 
	\item Quindi $m=1$ $q=2$ che sostituiti in $y=mx+q$ otteniamo l'equazione cercata $y=x+2$
\end{enumerate}
\begin{esempiot}{Dati due punti trovare l'equazione della retta}{retta6}
	Data la coppia di punti	$A\coord{-2}{3}$\,$B\coord{3}{-1}$ trovare l'equazione della retta che passa per questi punti.
\end{esempiot}
\begin{enumerate}
	\item Considero la retta generica $y=mx+q$
	\item Passaggio per $A\coord{-2}{3}$ ottengo $3=-2\cdot m+q$
	\item Passaggio per $B\coord{3}{-1}$ ottengo $-1=3\cdot m+q$
	\item Allineo i due risultati e sottraggo
	\begin{tabular}{rll}
		$+3=$&$-2\cdot m$ &$+q$  \\ 
		$-1=$&$+3\cdot m$ &$+q$  \\ 
		\hline $+4=$&$-5\cdot m$& $0$ \\ 
	\end{tabular} 
	\item Semplifico e ottengo $m=-\dfrac{4}{5}$
	\item Prendo una delle precedenti relazioni e sostituisco il valore di $m=-\dfrac{4}{5}$ trovato
	\item \begin{tabular}{rll}
		$-1=$&$+3\cdot(-\dfrac{4}{5}) $ &$+q$  \\ 
		$-1=$&$-\dfrac{12}{5}$ &$+q$  \\ 
		$-1+\dfrac{12}{5}=$& $+q$  \\ 
		$q=$&$\dfrac{7}{5}$ 
	\end{tabular} 
	\item Quindi $m=-\dfrac{4}{5}$ $q=\dfrac{7}{5}$ che sostituiti in $y=mx+q$ otteniamo l'equazione cercata $y=-\dfrac{4}{5}x+\dfrac{7}{5}$
\end{enumerate}
\section{Retta per un punto parallela a retta data}
\begin{esempiot}{Dato un punto e una retta parallela}{retta7}
	Data la retta $y=3x+4$ Trovare la retta parallela alla retta data che  passa per il punto	$A\coord{2}{3}$
\end{esempiot}
\begin{enumerate}
	\item Considero la retta generica $y=mx+q$
	\item Le due rette sono parallele quindi $m=3$ ottengo $y=3x+q$
	\item Passaggio per $A\coord{2}{3}$ ottengo 
	\begin{tabular}{rll}
	$3=$&$3\cdot 2$&$+q$\\
	$3-6=$&$q$\\
	$q=$&$-3$\\
	\end{tabular}
\end{enumerate}

La retta cercata è $y=3x-3$ procedendo come con l'\cref{exa:retta1}
otteniamo
\begin{center}
	\includestandalone[width=.6\textwidth]{terzo/grafici/retta_dis_7}
	\captionof{figure}{Retta parallela a retta data}\label{fig:disegnoretta7}
\end{center}
\begin{esempiot}{Dato un punto e una retta parallela}{retta8}
	Data la retta $y=-3x+5$ Trovare la retta parallela alla retta data che  passa per il punto	$A\coord{-2}{4}$
\end{esempiot}
\begin{enumerate}
	\item Considero la retta generica $y=mx+q$
	\item Le due rette sono parallele quindi $m=-3$ ottengo $y=-3x+q$
	\item Passaggio per $A\coord{-2}{4}$ ottengo 
	\begin{tabular}{rll}
		$4=$&$-3\cdot (-2)$&$+q$\\
		$4-6=$&$q$\\
		$q=$&$-2$\\
	\end{tabular}
\end{enumerate}

La retta cercata è $y=-3x-2$ procedendo come con l'\cref{exa:retta1}
otteniamo
\begin{center}
	\includestandalone[width=.6\textwidth]{terzo/grafici/retta_dis_8}
	\captionof{figure}{Retta parallela a retta data}\label{fig:disegnoretta8}
\end{center}
\begin{esempiot}{Dato un punto e una retta parallela}{retta8a}
	Trovare la retta parallela all'asse $x$ che passa per $A\coord{1}{2}$
\end{esempiot}
La retta cercata è $y=2$ procedendo come con l'\cref{exa:retta3}
otteniamo
\begin{center}
	\includestandalone[width=.6\textwidth]{terzo/grafici/retta_dis_8a}
	\captionof{figure}{Retta parallela all'asse $x$}\label{fig:disegnoretta8a}
\end{center}
\section{Retta per un punto perpendicolare a retta data}
\begin{esempiot}{Dato un punto e una retta perpendicolare}{retta9}
	Data la retta $y=2x+2$ Trovare la retta perpendicolare alla retta data che  passa per il punto	$A\coord{2}{4}$
\end{esempiot}
\begin{enumerate}
	\item Considero la retta generica $y=m_2x+q$
	\item Le due rette sono perpendicolari quindi $2\cdot m_2=-1$ ottengo $m_2=-\dfrac{1}{2}$ quindi $y=-\dfrac{1}{2}x+q$
	\item Passaggio per $A\coord{2}{4}$ ottengo 
	\begin{tabular}{rll}
		$4=$&$-\dfrac{1}{2}\cdot 2$&$+q$\\
		$4+1=$&$q$\\
		$q=$&$5$\\
	\end{tabular}
\end{enumerate}

La retta cercata è $y=-\dfrac{1}{2}x+5$ procedendo come con l'\cref{exa:retta3}
otteniamo
\begin{center}
	\includestandalone[width=.6\textwidth]{terzo/grafici/retta_dis_9}
	\captionof{figure}{Retta perpendicolare a retta data}\label{fig:disegnoretta9}
\end{center}
\begin{esempiot}{Dato un punto e una retta perpendicolare}{retta10}
	Data la retta $y=-x+3$ Trovare la retta perpendicolare alla retta data che  passa per il punto	$A\coord{-1}{-1}$
\end{esempiot}
\begin{enumerate}
	\item Considero la retta generica $y=m_2x+q$
	\item Le due rette sono perpendicolari quindi $-1\cdot m_2=-1$ ottengo $m_2=1$ quindi $y=x+q$
	\item Passaggio per $A\coord{-1}{-1}$ ottengo 
	\begin{tabular}{rll}
		$-1=$&$-1$&$+q$\\
		$-1+1=$&$q$\\
		$q=$&$0$\\
	\end{tabular}
\end{enumerate}

La retta cercata è $y=x$ procedendo come con l'\cref{exa:retta1}
otteniamo
\begin{center}
	\includestandalone[width=.6\textwidth]{terzo/grafici/retta_dis_10}
	\captionof{figure}{Retta perpendicolare a retta data}\label{fig:disegnoretta10}
\end{center}
\section{Fascio di rette}
\begin{esempiot}{Trovare l'equazione del fascio passante per un punto}{retta11}
	Dato il punto	$A\coord{3}{4}$ trovare l'equazione del fascio di centro $A$
\end{esempiot}
Partendo da $y-y_1=m(x-x_1)$ passaggio per 	$A\coord{3}{4}$
\begin{align*}
	y-4=&m(x-3)\\
	y=&m(x-3)+4
\end{align*}
L'equazione cercata è $y=m(x-3)+4$
\begin{esempiot}{Trovare l'equazione del fascio passante per un punto}{retta12}
	Dato il punto	$A\coord{3}{4}$ trovare l'equazione del fascio di centro $A$
\end{esempiot}
Partendo da $y-y_1=m(x-x_1)$ passaggio per 	$A\coord{-1}{-2}$
\begin{align*}
	y+2=&m(x+1)\\
	y=&m(x+1)-2
\end{align*}
L'equazione cercata è $y=m(x+1)$
\subsection{Retta per un punto parallela a retta data}
\begin{esempiot}{Dato un punto e una retta parallela}{retta13}
	Data la retta $y=3x+4$ Trovare la retta parallela alla retta data che  passa per il punto	$A\coord{2}{3}$
\end{esempiot}
Considero l'equazione generica del fascio $y-y_1=m(x-x_1)$, se le due rette sono parallele $m=3$ quindi
\begin{align*}
	y-y_1=&3(x-x_1)\\
	\intertext{Passaggio per $A\coord{2}{3}$}
	y-3=&3(x-2)\\
	y=&3x-6+3\\
	y=&3x-3\\
\end{align*}
Otteniamo 	$y=3x-3$ lo stesso risultato dell'\cref{exa:retta7}
\subsection{Retta passante per un punto perpendicolare a retta data}
\begin{esempiot}{Dato un punto e una retta perpendicolare}{retta14}
	Data la retta $y=2x+2$ Trovare la retta perpendicolare alla retta data che  passa per il punto	$A\coord{2}{4}$
\end{esempiot}
Considero l'equazione generica del fascio $y-y_1=m(x-x_1)$, se le due rette sono parallele $m_1\cdot m_2=-1$ quindi $2m_2=-1$ $m_2=-\dfrac{1}{2}$
\begin{align*}
	y-y_1=&-\dfrac{1}{2}(x-x_1)\\
	\intertext{Passaggio per $A\coord{2}{4}$}
	y-4=&-\dfrac{1}{2}(x-2)\\
	y=&-\dfrac{1}{2}x+1+4\\
	y=&-\dfrac{1}{2}x+5\\
\end{align*}
Otteniamo 	$y=-\dfrac{1}{2}x+5$ lo stesso risultato dell'\cref{exa:retta9}
\section{Intersezioni fra rette}
\begin{esempiot}{Date due rette trovare il punto di intersezione}{retta15}
	Data le rette $y=-2x+4$ e $y=x+1$ trovare il loro eventuale punto di intersezione
\end{esempiot}
Imposto il sistema 
\[\begin{cases} 
	y=-2x+4\\
	y=+x+1
\end{cases}\]
Risolvo il sistema formato dalle due rette in forma esplicita con il metodo del confronto.
\begin{align*}
	&\begin{cases} 
		y=-2x+4\\
		y=+x+1
	\end{cases}&&\begin{cases} 
	x+1=-2x+4\\
	y=+x+1
\end{cases}\\
&\begin{cases} 
	3x=3\\
	y=x+1
\end{cases}&&\begin{cases} 
x=1\\
y=x+1
\end{cases}\\
&\begin{cases} 
	x=1\\
	y=1+1
\end{cases}&&
\begin{cases} 
	x=1\\
	y=2
\end{cases}
\end{align*}
Le due rette si incontrano in $A\coord{1}{2}$
\begin{esempiot}{Date due rette trovare il punto di intersezione}{retta16}
	Data le rette $y=\dfrac{1}{2}x+4$ e $y=-x+1$ trovare il loro eventuale punto di intersezione
\end{esempiot}
Imposto il sistema 
\[\begin{cases} 
y=+\dfrac{1}{2}x+4\\
y=-x+1
\end{cases}\]
Risolvo il sistema formato dalle due rette in forma esplicita con il metodo del confronto.
\begin{align*}
	&\begin{cases} 
		y=\dfrac{1}{2}x+4\\
		y=-x+1
	\end{cases}&&\begin{cases} 
	-x+1=\dfrac{1}{2}x+4\\
	y=-x+1
\end{cases}\\
&\begin{cases} 
	-2x+2=+x+8\\
	y=-x+1
\end{cases}&&\begin{cases} 
-2x-x=+8-2\\
y=-x+1
\end{cases}\\
&\begin{cases} 
	-3x=+6\\
	y=-x+1
\end{cases}&&
\begin{cases} 
	x=-2\\
	y=-x+1
\end{cases}\\
&\begin{cases} 
	x=-2\\
	y=3
\end{cases}
\end{align*}
Le due rette si incontrano in $A\coord{-2}{3}$
\begin{cesempiot}{Date due rette trovare il punto di intersezione}{retta17}
	Data le rette $y=2x+5$ e $y=2x+6$ trovare il loro eventuale punto di intersezione
\end{cesempiot}
Imposto il sistema 
\[\begin{cases} 
y=2x+5\\
y=2x+6
\end{cases}\]
Risolvo il sistema formato dalle due rette in forma esplicita con il metodo del confronto.
\begin{align*}
	&\begin{cases} 
		y=2x+5\\
		y=2x+6
	\end{cases}&&\begin{cases} 
	2x+5=2x+6\\
	y=2x+6
\end{cases}\\
&\begin{cases} 
	5=6\\
	y=2x+6
\end{cases}
\end{align*}
Le due rette non si intersecano.